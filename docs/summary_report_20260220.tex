\documentclass[a4paper,11pt,dvipdfmx]{jsarticle}

\usepackage[top=20mm, bottom=25mm, left=20mm, right=20mm]{geometry}
\usepackage{url}
\usepackage{graphicx}
\usepackage{amssymb}
\usepackage{amsmath}
\usepackage{booktabs}
\usepackage{float}
\usepackage{multirow}
\usepackage{tikz}
\usetikzlibrary{shapes,arrows,positioning}

\title{ローリング計画法を用いたPV・蓄電池システムの\\運用最適化と電気料金削減効果\\[0.5em]\large 月次進捗報告}
\author{神戸大学工学部情報知能工学科4年\\山崎博之}
\date{2026年2月20日}

\begin{document}

\maketitle

\section{今月の取り組み概要}

今月は以下の3つの実験・分析を実施した.

\begin{enumerate}
    \item 年間一括最適化:年間全データ(17,520ステップ)を一括で最適化し,ローリング計画法の結果と比較
    \item 再計画間隔の延長:ローリング計画法において,再計画間隔$m$を1ステップ(30分)から8ステップ(4時間)に拡大し,計算効率への影響を検証
    \item 長期予測期間の検証:予測期間$H$を48時間から最大21日間(1,008ステップ)まで延長し,市場価格連動プランの経済性改善を検証
\end{enumerate}

\section{年間一括最適化の結果}

\subsection{実装概要}

年間一括最適化では,ローリング計画法と同一の目的関数(式(\ref{eq:objective}))・制約条件を用いつつ,予測期間$H$を年間全ステップ($H=17{,}520$)に設定した.各プランにおいて2~3分程度で最適解が得られた.

\begin{equation}
\text{Minimize} \quad J = \underbrace{w_{\mathrm{basic}} \cdot s^{\mathrm{BY}}_{\mathrm{MAX}}}_{\text{基本料金}} + \underbrace{\sum_{k=0}^{H-1} p^{\mathrm{BY}}_{k} \cdot s^{\mathrm{BY}}_{k} \cdot \Delta t}_{\text{電力量料金}}
\label{eq:objective}
\end{equation}

\subsection{ローリング計画法との比較}

年間全データを一括で最適化し,ローリング計画法(48時間予測,$m=1$)と比較した結果を表~\ref{tab:rolling_vs_annual}に示す.

\begin{table}[H]
\centering
\caption{ローリング計画法(48時間予測)と年間一括最適化の比較(蓄電池容量860\,kWh)}
\label{tab:rolling_vs_annual}
\begin{tabular}{lrrrr}
\toprule
手法 & 北電基本 [万円] & 市場連動 [万円] & 計算時間 [分] \\
\midrule
ローリング計画法(48時間予測,$m=1$) & 1,434.4 & 1,502.4 & 33.2 \\
年間一括最適化 & 1,427.8 & 1,262.9 & 4.5 \\
\midrule
差額 & 6.6 (0.5\%) & 239.5 (15.9\%) & 28.7 \\
\bottomrule
\end{tabular}
\end{table}

\noindent
※表中の「差額」は各プランにおける(年間一括最適化の電気料金 $-$ ローリング計画法の電気料金)を示す.

\noindent
\begin{itemize}
    \item 北海道電力基本プランでは,ローリング計画法でもほぼ最適解(差0.5\%)を達成
    \item 市場価格連動プランでは,ローリング計画法と年間一括最適化で約240万円(15.9\%)の差
    \item 年間一括最適化の計算時間は約4.5分と,ローリング計画法(33分)より約7倍高速
\end{itemize}

\subsection{料金プラン優位性の逆転}

年間一括最適化の結果に基づき,料金プランの優位性を再評価した(表~\ref{tab:plan_reversal}).

\begin{table}[H]
\centering
\caption{料金プランの比較:ローリング計画法と年間一括最適化}
\label{tab:plan_reversal}
\begin{tabular}{lrrl}
\toprule
手法 & 北電基本 [万円] & 市場連動 [万円] & 安価なプラン \\
\midrule
ローリング(48時間) & 1,434.4 & 1,502.4 & 北電基本 \\
年間一括最適化 & 1,427.8 & 1,262.9 & 市場連動 \\
\bottomrule
\end{tabular}
\end{table}

\noindent
年間一括最適化では市場価格連動プランが約165万円安価であり,48時間ローリング計画での結果「北電基本プランが有利」と逆転する.これは,ローリング計画法の短い予測期間(48時間)において市場価格連動プランが局所最適に陥っていることを示している.

\section{再計画間隔の影響}

再計画間隔$m$とは,1回の最適化計算で実際に採用・実行する決定ステップ数である.$m$を増加させることで計画頻度を下げ計算負荷を削減できるが,予測誤差への対応が遅れるというトレードオフがある.

\begin{table}[H]
\centering
\caption{再計画間隔の影響(予測期間48時間,蓄電池容量860\,kWh)}
\label{tab:control_horizon}
\begin{tabular}{lrrr}
\toprule
再計画間隔 & 北電基本 [万円] & 市場連動 [万円] & 計算時間 [分] \\
\midrule
1ステップ(30分) & 1,434.4 & 1,502.4 & 33.2 \\
2ステップ(1時間) & 1,434.4 & 1,502.2 & 15.9 \\
4ステップ(2時間) & 1,434.4 & 1,501.6 & 7.9 \\
8ステップ(4時間) & 1,434.4 & 1,500.7 & 4.1 \\
\bottomrule
\end{tabular}
\end{table}

\noindent
\begin{itemize}
    \item 再計画間隔$m=8$で計算時間を88\%削減(33分→4分)
    \item 北海道電力基本プランでは電気料金への影響なし,市場価格連動プランでも最大0.1\%(約0.2万円)と極めて限定的
    \item 計算効率の大幅改善と解品質の維持を両立できることを確認
\end{itemize}

\section{長期予測期間の効果}

再計画間隔の延長により計算時間を抑制しつつ,予測期間$H$を最大21日間まで延長した結果を表~\ref{tab:long_horizon}に示す.

\begin{table}[H]
\centering
\small
\caption{長期予測期間と再計画間隔の影響(蓄電池容量860\,kWh)}
\label{tab:long_horizon}
\begin{tabular}{llrrrr}
\toprule
予測期間 & 再計画間隔 $m$ & 北電基本 [万円] & 市場連動 [万円] & 計算時間 [分] & コスト増加率 \\
\midrule
48時間(96) & 1 & 1,434.4 & 1,502.4 & 33.2 & 19.0\% \\
\midrule
4日間(192) & 4 & 1,433.1 & 1,452.5 & 23.1 & 15.0\% \\
4日間(192) & 8 & 1,433.1 & 1,448.3 & 11.1 & 14.7\% \\
\midrule
8日間(384) & 4 & 1,431.5 & 1,420.9 & 53.1 & 12.5\% \\
8日間(384) & 8 & 1,431.5 & 1,420.6 & 25.5 & 12.5\% \\
\midrule
14日間(672) & 8 & 1,430.4 & 1,398.9 & 48.5 & 10.8\% \\
14日間(672) & 16 & 1,430.4 & 1,398.8 & 25.1 & 10.8\% \\
\midrule
21日間(1008) & 24 & 1,429.9 & 1,380.2 & 27.6 & 9.3\% \\
\midrule
\multicolumn{2}{l}{年間一括最適化} & 1,427.8 & 1,262.9 & 4.5 & 0\% \\
\bottomrule
\end{tabular}
\end{table}

\noindent
※「コスト増加率」は、年間一括最適化によって得られる電気料金に対し、ローリング計画法によって生じた過大コストの割合を示す。
\begin{equation}
    \text{コスト増加率} = \frac{\text{ローリング計画法の電気料金} - \text{年間一括最適化の電気料金}}{\text{年間一括最適化の電気料金}} \times 100
\end{equation}

\noindent
主な発見:
\begin{itemize}
    \item 14日間予測で市場連動プランの電気料金が1,398.9万円まで改善(48時間比:103.5万円の削減)
    \item 年間一括最適化と比べたコスト増加率が19.0\%から10.8\%に縮小
    \item 8日間以上の予測期間で市場連動プランの優位性が回復
    \item 再計画間隔$m=8$を用いれば14日間予測でも計算時間は約48.5分に抑制可能
\end{itemize}

\subsection{長期予測期間と蓄電池容量の組み合わせ}

8日間予測($m=8$)において,蓄電池容量を変化させた結果を表~\ref{tab:battery_long}に示す.

\begin{table}[H]
\centering
\caption{8日間予測($m=8$)における蓄電池容量の比較}
\label{tab:battery_long}
\begin{tabular}{rrrrrl}
\toprule
容量 [kWh] & 北電基本 [万円] & 市場連動 [万円] & 差額 [万円] & 計算時間 [分] & 安価なプラン \\
\midrule
430 & 1,486.4 & 1,402.6 & $+$83.8 & 105.3 & 市場連動 \\
860 & 1,431.5 & 1,420.6 & $+$10.9 & 25.5 & 市場連動 \\
1290 & 1,402.9 & 1,452.7 & $-$49.8 & 35.2 & 北電基本 \\
1720 & 1,375.6 & 1,452.5 & $-$76.9 & 37.5 & 北電基本 \\
\bottomrule
\end{tabular}
\end{table}

\noindent
※表中の「差額」は(北電基本プランの電気料金 $-$ 市場連動プランの電気料金)を示す.値がプラスである場合は市場連動プランの方が安価であることを意味する.

\noindent
\begin{itemize}
    \item 8日間予測では,蓄電池容量860kWh以下で市場連動プランが有利
    \item 48時間予測では540kWh以上で北電基本プランが有利だったが,長期予測によりプラン優位性の分岐点が860kWh付近まで移動
    \item 860kWhで市場連動プランの契約電力が193.7kW(48時間予測の218.1kWから24.4kW減少)
\end{itemize}

\section{まとめ}

\begin{enumerate}
    \item 年間一括最適化の結果と、ローリング計画法の結果を比較した.北海道電力基本プランではローリング計画法でもほぼ最適解(差0.5\%)を達成するが,市場価格連動プランでは15.9\%の差が生まれた.
    \item 48時間ローリングでは540kWh以上で北電基本プランが有利となるのに対し、年間一括最適化では全ての蓄電池容量で市場連動プランが有利であるという結果となった.
    \item 再計画間隔を4時間に延長することで計算時間を88\%削減しつつ,電気料金影響0.1\%未満に抑えられることを確認した.
    \item 予測期間を14日間にすることで、年間一括最適化に対するコスト増加率を、予測期間を2日間にした場合よりも減少(19.0\%$\to$10.8\%)させた.
\end{enumerate}

\end{document}
