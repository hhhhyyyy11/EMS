\documentclass[a4paper,12pt,dvipdfmx]{jsarticle}

\usepackage[top=20mm, bottom=25mm, left=20mm, right=20mm]{geometry}
\usepackage{url}
\usepackage{graphicx}
\usepackage{amssymb}
\usepackage{amsmath}
\usepackage{booktabs}
\usepackage{float}
\usepackage{multirow}
\usepackage{tikz}
\usetikzlibrary{shapes,arrows,positioning}

\title{ローリング計画法を用いたPV・蓄電池システムの\\運用最適化と電気料金削減効果の分析}
\author{神戸大学工学部情報知能工学科4年\\山崎博之}
\date{2026年2月19日}

\begin{document}

\maketitle

\section{背景と目的}

本研究では,北海道十勝地方に設置された出力250kWの太陽光発電(PV)・蓄電池システムを対象に,ローリング計画法を用いた年間電気料金の最小化について検討する.

シミュレーションには,2024年1月1日から12月31日までの365日間(2月29日を除く)における施設の電力消費量,PV発電量,および日本卸電力取引所(JEPX)のスポット価格の実測データ(30分間隔,計17,520ステップ)を用いた.料金体系として,北海道電力の基本プラン(高圧電力・固定料金)と市場価格連動プランの2種類を設定し,蓄電池容量を0kWhから1720kWhの範囲で変動させて比較を行った.最適化手法には,充放電の排他制御に二値変数が必要であり,かつ制約条件が線形で表現できることから,混合整数線形計画法(MILP)を採用した.ソルバーにはオープンソースで高性能なPySCIPOptを用い,予測期間(ホライズン)は基準として96ステップ(48時間)を設定し,24時間から72時間の範囲で比較検証を行った.
% 30分間隔を60分間隔でやってみる検討をする
本研究の主たる目的は以下の4点である:
\begin{enumerate}
    \item \textbf{運用モデルの構築と実証}:PV・蓄電池システムに対するローリング計画法を構築・実装し,実データを用いたシミュレーションを通じて年間電気料金の最小化を図る.
    \item \textbf{蓄電池容量と料金プランの経済性評価}:蓄電池容量を変化させながら,固定料金プランと市場価格連動プランの経済性を比較する.特に,各プランにとって最適な蓄電池容量を導出し,その条件下での経済的優位性を検証する.
    \item \textbf{予測期間の影響分析}:ローリング計画法における予測期間を24時間,48時間,72時間と変化させ,予測期間長が蓄電池の充放電運用および経済性に及ぼす影響を定量的に分析する.
    \item \textbf{年間一括最適化との比較}:ローリング計画法の理論的限界を評価するため,年間全ステップを一括で最適化し,ローリング計画法による解との差を定量的に評価する.
\end{enumerate}

\section{システム構成と制約条件}

\subsection{用語の定義}

本論文で使用する主要な用語を以下に定義する.

\begin{itemize}
    \item \textbf{電力系統(系統)}:発電所から電力消費者(施設や家庭など)まで電力を供給するための送電・配電網の総称.本研究では,施設が電力会社から電力を購入する際の接続先を指す.
    \item \textbf{買電}:電力系統から電力を購入すること.その電力量を買電量 [kWh],瞬時電力を買電電力 [kW] と呼ぶ.
    \item \textbf{逆潮流}:施設側から電力系統へ電力を送り返すこと(売電).本研究では逆潮流を禁止する設定とした.
    \item \textbf{契約電力}:電力会社との契約で定める最大需要電力 [kW].高圧契約の場合,過去1年間の各月の最大需要電力(30分平均値)のうち最大のものが適用される.基本料金はこの契約電力に比例する.
    \item \textbf{出力抑制}:PV発電可能量の一部を意図的に使用せず捨てること.蓄電池が満充電かつ需要が少ない場合などに発生する.
\end{itemize}

\subsection{対象システムの概要}

本研究では,2024年1月1日から同年12月31日までの365日間(2月29日を除く17,520ステップ,30分時間解像度)を対象期間とし,以下の仕様を持つシステムについてシミュレーションを行った.

\begin{itemize}
    \item \textbf{太陽光発電(PV)システム}:定格出力は250kWであり,パネルは南向き,設置角度40°で固定されている.
    \item \textbf{蓄電池システム}:本研究では蓄電池容量を0kWhから1720kWhの範囲で変化させて経済性を比較するが,基準容量として860kWh(430kWh$\times$2基)を用いる.最大充放電出力は400kWである.初期SOC(State of Charge:蓄電池の充電残量を0〜100\%で表す指標)は容量の50\%とし,充放電効率はいずれも0.98(リチウムイオン電池の一般的な値)と設定した.
    \item \textbf{系統電力}:買電単価は市場価格連動(JEPX)または固定料金を適用する.基本料金単価は北海道電力の高圧電力料金(2,829.60円/kW)に基づき,力率割引(85\%)を考慮した値を採用した.なお,本システムは完全自家消費型であり,逆潮流は行わない設定とした.
\end{itemize}

システム構成の概念図を図\ref{fig:system_config}に示す.

\begin{figure}[H]
\centering
\begin{tikzpicture}[
    node distance=1.5cm,
    block/.style={rectangle, draw, minimum width=2.5cm, minimum height=1cm, align=center},
    arrow/.style={->, thick},
    darrow/.style={<->, thick}
]
% ノード定義
\node[block] (grid) {電力系統\\(北電/JEPX)};
\node[block, right=2.5cm of grid] (load) {施設負荷\\(需要)};
\node[block, above=1.2cm of load] (pv) {太陽光発電\\(PV: 250kW)};
\node[block, below=1.2cm of load] (battery) {蓄電池\\(860kWh)};

% 矢印
\draw[arrow] (grid) -- node[above] {$s^{\mathrm{BY}}_k$} node[below] {買電} (load);
\draw[arrow] (pv) -- node[right] {$g^{\mathrm{P2}}_k$} (load);
\draw[darrow] (battery) -- node[right] {$x^{\mathrm{FC}}_k / x^{\mathrm{FD}}_k$} node[left] {充放電} (load);

% 注釈
\node[below=0.3cm of battery, text width=3cm, align=center, font=\small] {SOC: $b^{\mathrm{F}}_k$};
\end{tikzpicture}
\caption{システム構成の概念図.電力系統からの買電,PV発電,蓄電池の充放電により施設需要を満たす.}
\label{fig:system_config}
\end{figure}

\subsection{運用制約条件}

システム運用における最適化計算では,以下の物理的および制度的制約を課した.

\begin{itemize}
    \item \textbf{電力需給平衡}:各タイムステップにおいて,供給電力(PV発電,買電,蓄電池放電)の総和は,需要電力(施設需要,蓄電池充電)の総和と常に一致しなければならない.
    \item \textbf{蓄電池運用制約}:蓄電池の劣化抑制および安全性を考慮し,SOCの運用範囲は定格容量の5\%から95\%に制限した(容量860kWhの場合,43kWh以上817kWh以下).また,充放電電力は最大出力(400kW)以下とし,充電と放電の同時実行を禁止する排他制御も行う.
    \item \textbf{逆潮流禁止制約}:PV発電の余剰電力が生じた場合でも,売電量は常に0とし,システム内で消費または出力抑制するものとした.
\end{itemize}


\section{最適化手法}

本研究では,不確実性を伴う環境下での運用計画を立案するため,ローリング計画法(Rolling Horizon Approach)を用いた混合整数線形計画法(MILP: Mixed Integer Linear Programming)を採用する.

\subsection{ローリング計画法の適用}

時間軸を$\Delta t = 0.5$時間(30分)間隔で離散化し,各時刻をステップ$k \in \mathbb{Z}_{\geq 0}$で表現する.ローリング計画法では,現在時刻$k$から予測期間(ホライズン)$H$ステップ先までの最適化問題を解き,得られた解のうち直近の第1ステップ(時刻$k$の制御入力)のみを実行する.次ステップ$k+1$においては,最新のシステム状態(蓄電池SOC等)および予測されるPV発電量・需要に基づき,再び$H$ステップ先までの計画を更新する.
本手法の採用により,年間を通した大域的な最適化(全17,520ステップを一括で解く場合)に伴う膨大な計算負荷を,各ステップで$H$ステップ分の小規模な問題に分割することで実用的な範囲に低減できる.また,逐次的に計画を更新するため,予測誤差の影響を最小限に抑えたフィードバック制御的な運用が可能となる.本研究では,予測期間を$H=96$(48時間)を基準として設定した.

\subsection{料金体系}

本研究で比較検討する2種類の料金体系について説明する.

\subsubsection{北海道電力基本プラン(高圧電力,一般料金)}

北海道電力の料金体系を表\ref{tab:hokkaido_tariff}に示す.

\begin{table}[H]
\centering
\caption{北海道電力の料金体系(2024年4月1日実施)}
\label{tab:hokkaido_tariff}
\begin{tabular}{lc}
\toprule
項目 & 料金単価 \\
\midrule
基本料金 & 2,829.60 円/kW \\
電力量料金 & 21.51 円/kWh \\
再エネ賦課金 & 3.98 円/kWh \\
\bottomrule
\end{tabular}
\end{table}

\textbf{基本料金の計算式:}
\begin{equation}
C_{\mathrm{basic}} = P_{\mathrm{contract}} \times 2829.60 \times 0.85 \times 12 \quad \text{[円/年]}
\end{equation}

ここで,$P_{\mathrm{contract}}$は契約電力であり,過去1年間の各月の最大需要電力のうち,最も大きい値を適用する.2829.60は基本料金単価(円/kW),0.85は力率割引係数,12は年間月数である.本シミュレーションでは,1年間の運用結果から得られた最大買電電力を$P_{\mathrm{contract}}$として事後的に計算している.

\textbf{電力量料金の計算式:}
\begin{equation}
C_{\mathrm{energy}} = E_{\mathrm{month}} \times (21.51 + F_{\mathrm{adj}}(m) + 3.98) \quad \text{[円/月]}
\end{equation}

ここで,$E_{\mathrm{month}}$は月間電力使用量 [kWh],21.51は電力量料金単価(円/kWh),$F_{\mathrm{adj}}(m)$は$m$月($m \in \{1,2,\dots,12\}$)の燃料費調整額(円/kWh),3.98は再生可能エネルギー発電促進賦課金(円/kWh)である.

2024年の月別燃料費調整額を表\ref{tab:fuel_adjustment}に示す.

\begin{table}[H]
\centering
\caption{2024年の月別燃料費調整額(北海道電力・高圧)}
\label{tab:fuel_adjustment}
\begin{tabular}{cc}
\toprule
月 & 燃料費調整額 [円/kWh] \\
\midrule
1月 & $-8.76$ \\
2月 & $-8.59$ \\
3月 & $-8.56$ \\
4月 & $-8.85$ \\
5月 & $-9.02$ \\
6月 & $-7.47$ \\
7月 & $-5.69$ \\
8月 & $-5.69$ \\
9月 & $-9.60$ \\
10月 & $-9.47$ \\
11月 & $-8.06$ \\
12月 & $-5.83$ \\
\bottomrule
\end{tabular}
\end{table}

\subsubsection{市場価格連動プラン}

市場価格連動プランでは,電力量料金がJEPX(日本卸電力取引所)のスポット価格に連動する.

\textbf{電力量料金の計算式:}
\begin{equation}
C_{\mathrm{energy}} = E_{\mathrm{month}} \times (P_{\mathrm{JEPX}}(t) + 3.98) \quad \text{[円/月]}
\end{equation}

ここで,$P_{\mathrm{JEPX}}(t)$は時刻 $t$ のJEPXスポット価格(円/kWh),3.98は再生可能エネルギー発電促進賦課金(円/kWh)である.基本料金は北海道電力と同額とする.

\subsection{数理モデルの定式化}

本節で使用する記号を表\ref{tab:symbols}にまとめる.

\begin{table}[H]
\centering
\caption{記号一覧}
\label{tab:symbols}
\begin{tabular}{clc}
\toprule
記号 & 説明 & 単位 \\
\midrule
\multicolumn{3}{l}{\textbf{決定変数}} \\
$s^{\mathrm{BY}}_{k}$ & 時刻$k$の買電電力 & kW \\
$s^{\mathrm{BY}}_{\mathrm{MAX}}$ & 予測期間内の最大買電電力 & kW \\
$x^{\mathrm{FC1}}_{k}$, $x^{\mathrm{FC2}}_{k}$ & 充電電力(変換前/後) & kW \\
$x^{\mathrm{FD1}}_{k}$, $x^{\mathrm{FD2}}_{k}$ & 放電電力(変換前/後) & kW \\
$b^{\mathrm{F}}_{k}$ & 蓄電池残量(SOC) & kWh \\
$z_{k}$ & 充放電モード(1:充電, 0:放電) & -- \\
$g^{\mathrm{P2}}_{k}$ & 実際に使用するPV発電量 & kW \\
\midrule
\multicolumn{3}{l}{\textbf{既知パラメータ}} \\
$g^{\mathrm{P1}}_{k}$ & PV発電可能量 & kW \\
$d_{k}$ & 需要電力 & kW \\
$p^{\mathrm{BY}}_{k}$ & 電力量単価 & 円/kWh \\
\midrule
\multicolumn{3}{l}{\textbf{定数}} \\
$H$ & 予測期間(ホライズン) & ステップ \\
$\Delta t$ & 1ステップあたりの時間(= 0.5) & h \\
$\eta$ & 充放電効率(= 0.98) & -- \\
$C_{\mathrm{bat}}$ & 蓄電池定格容量 & kWh \\
$P^{\mathrm{FC}}_{\mathrm{max}}$, $P^{\mathrm{FD}}_{\mathrm{max}}$ & 最大充放電出力(= 400) & kW \\
$C_{\mathrm{cap}}$ & 年間基本料金単価 & 円/kW \\
$w_{\mathrm{basic}}$ & 基本料金重み係数 & 円/kW \\
$M$ & Big-M定数(= $10^6$) & -- \\
\bottomrule
\end{tabular}
\end{table}

\subsubsection{決定変数}

最適化問題における決定変数は表\ref{tab:symbols}の通りである.連続変数は非負とし,添字$k$は時刻ステップ($k = 0, 1, \ldots, H-1$)を表す.

既知パラメータとして,PV発電可能量$g^{\mathrm{P1}}_{k}$および需要電力$d_{k}$を与える.出力抑制を許容し,$g^{\mathrm{P2}}_{k} \leq g^{\mathrm{P1}}_{k}$とする.

\subsubsection{目的関数}

日本の高圧電力契約では,電気料金は以下の2つから構成される:
\begin{itemize}
    \item \textbf{基本料金}:契約電力(過去1年間の最大需要電力)に基づく固定費
    \item \textbf{電力量料金}:実際の使用電力量に応じた従量費
\end{itemize}

\noindent
目的関数は,予測期間$H$における電気料金(基本料金相当額および電力量料金)の総和とし,これを最小化する.

\begin{equation}
\text{Minimize} \quad J = \underbrace{w_{\mathrm{basic}} \cdot s^{\mathrm{BY}}_{\mathrm{MAX}}}_{\text{基本料金相当額}} + \underbrace{\sum_{k=0}^{H-1} p^{\mathrm{BY}}_{k} \cdot s^{\mathrm{BY}}_{k} \cdot \Delta t}_{\text{電力量料金}}
\label{eq:objective}
\end{equation}

\noindent
\textbf{目的関数の設計に関する考察:}
本研究の目的関数は各予測期間$H$内のコスト最小化を行うものであり,真に最小化すべき年間電気料金を直接の目的関数としていない点に留意が必要である.契約電力は過去1年間の最大需要電力で決定されるため,本来は全期間(17,520ステップ)を一括で最適化することが理想的である.しかし,計算負荷の観点から実用的ではないため,ローリング計画法を採用した.

この制約により,予測期間が短い場合には基本料金項の重みが相対的に小さくなり,契約電力削減のインセンティブが弱まる可能性がある.一方で,ローリング計画法は実運用において以下の利点を持つ:
\begin{itemize}
    \item 予測誤差の蓄積を防ぎ,最新情報に基づく柔軟な計画更新が可能
    \item 計算負荷を実用的な範囲に抑制
    \item 異常事態への即応性の確保
\end{itemize}
したがって,本手法は「どのプランが経済的に有利か」という評価目的よりも,「実運用環境下での最適制御戦略の導出」という目的に適している.プラン比較の観点では,本研究の結果は各プランの傾向を示す参考値として解釈されるべきである.

ここで,$p^{\mathrm{BY}}_{k}$は時刻$k$における電力量単価 [円/kWh],$w_{\mathrm{basic}}$は基本料金に関する重み係数 [円/kW]である.

基本料金は本来,年間の最大需要電力に基づいて決定されるが,本手法では予測期間が限定的であるため,年間の基本料金単価を時間按分した値をペナルティ項として導入した.重み係数$w_{\mathrm{basic}}$は式(\ref{eq:weight_basic})により算出される.

\begin{equation}
w_{\mathrm{basic}} = C_{\mathrm{cap}} \times \frac{H \cdot \Delta t}{8760}
\label{eq:weight_basic}
\end{equation}

ここで,8760は1年間の時間数($365 \times 24$時間)であり,年間基本料金を予測期間の長さに応じて按分している.

ただし,$C_{\mathrm{cap}}$は年間基本料金単価であり,表\ref{tab:hokkaido_tariff}の料金単価に基づき以下のように算出される:
\begin{equation}
C_{\mathrm{cap}} = 2829.60 \times 0.85 \times 12 \quad \text{[円/kW]}
\end{equation}
ここで,2829.60円/kWは基本料金単価,0.85は力率割引係数(力率85\%以上で適用),12は年間月数である.この項の導入により,各予測期間における買電電力の最大値の抑制を図り,間接的に年間契約電力の低減を指向する.

\noindent
\textbf{注}:本手法は「局所的な最大値の抑制」を通じて「大域的な最大値(年間契約電力)」を間接的に制御する近似であり,その限界については考察(\ref{sec:wbasic_limitation}節)で詳述する.

\subsubsection{制約条件}

システムの物理的特性および運用上の要請に基づき,以下の制約条件を課す.

\noindent
\textbf{(1) 電力需給バランス制約}

各時刻において,供給と需要は一致しなければならない.
\begin{equation}
g^{\mathrm{P2}}_{k} + s^{\mathrm{BY}}_{k} + x^{\mathrm{FD2}}_{k} = d_{k} + x^{\mathrm{FC1}}_{k}, \quad \forall k
\label{eq:balance}
\end{equation}

充放電電力は変換効率$\eta = 0.98$を介して以下の関係にある.
\begin{align}
x^{\mathrm{FC2}}_{k} &= \eta \cdot x^{\mathrm{FC1}}_{k} \label{eq:fc_eff} \\
x^{\mathrm{FD2}}_{k} &= \eta \cdot x^{\mathrm{FD1}}_{k} \label{eq:fd_eff}
\end{align}

ここで,$x^{\mathrm{FC1}}_{k}$は蓄電池への入力電力(変換前),$x^{\mathrm{FC2}}_{k}$は蓄電池に実際に蓄積される電力(変換後)を表す.放電についても同様に,$x^{\mathrm{FD1}}_{k}$は蓄電池から取り出す電力(変換前),$x^{\mathrm{FD2}}_{k}$は負荷で利用可能な電力(変換後)である.なお,本システムでは逆潮流を禁止(売電量ゼロ)とする.

\noindent
\textbf{(2) 蓄電池状態遷移および容量制約}

蓄電池のSOC推移は次式で記述される.
\begin{equation}
b^{\mathrm{F}}_{k+1} = b^{\mathrm{F}}_{k} + (x^{\mathrm{FC2}}_{k} - x^{\mathrm{FD1}}_{k}) \cdot \Delta t
\label{eq:soc_update}
\end{equation}

また,過充電・過放電防止のため,運用範囲を定格容量$C_{\mathrm{bat}}$の5\%〜95\%に制限する.
\begin{equation}
0.05 \cdot C_{\mathrm{bat}} \leq b^{\mathrm{F}}_{k} \leq 0.95 \cdot C_{\mathrm{bat}}, \quad \forall k
\label{eq:soc_limit}
\end{equation}

\noindent
\textbf{(3) 充放電排他および出力制約}

充電と放電の同時実行を物理的に排除するため,二値変数$z_{k}$を用いた以下の制約(Big-M法)を設ける.
\begin{align}
x^{\mathrm{FC1}}_{k} &\leq M \cdot z_{k}, \quad \forall k \label{eq:bigm_fc} \\
x^{\mathrm{FD1}}_{k} &\leq M \cdot (1 - z_{k}), \quad \forall k \label{eq:bigm_fd}
\end{align}

ここで,$M$は十分大きな定数($M = 10^6$)である.また,充放電出力の上限として以下を課す.
\begin{align}
x^{\mathrm{FC2}}_{k} &\leq P^{\mathrm{FC}}_{\mathrm{max}}, \quad \forall k \label{eq:fc_limit} \\
x^{\mathrm{FD1}}_{k} &\leq P^{\mathrm{FD}}_{\mathrm{max}}, \quad \forall k \label{eq:fd_limit}
\end{align}

本研究では$P^{\mathrm{FC}}_{\mathrm{max}} = P^{\mathrm{FD}}_{\mathrm{max}} = 400$\,kWとした.

\noindent
\textbf{(4) PV出力制約}

実際に使用するPV発電量は発電可能量を超えない.
\begin{equation}
g^{\mathrm{P2}}_{k} \leq g^{\mathrm{P1}}_{k}, \quad \forall k
\label{eq:pv_limit}
\end{equation}

\noindent
\textbf{(5) 契約電力制約}

予測期間内のすべての時刻において,買電電力は$s^{\mathrm{BY}}_{\mathrm{MAX}}$以下でなければならない.
\begin{equation}
s^{\mathrm{BY}}_{k} \leq s^{\mathrm{BY}}_{\mathrm{MAX}}, \quad \forall k
\label{eq:contract}
\end{equation}

\section{実験設定}

\subsection{使用データ}

本シミュレーションでは,以下の実測データおよび市場価格データを使用した.

\begin{itemize}
    \item \textbf{対象期間}:2024年1月1日から同年12月31日までの365日間.2024年は閏年であるが,データの整合性を考慮し2月29日を除外した全17,520ステップ(30分時間分解能)を解析対象とした.
    \item \textbf{電力需要およびPV発電量}:北海道十勝地方の対象施設における実測値(30分積算値)を使用した.
    \item \textbf{電力市場価格}:一般社団法人日本卸電力取引所(JEPX)が公開する北海道エリアのスポット市場価格(30分値)を採用した.
\end{itemize}

\subsection{データ前処理}

取得データは30分間の積算電力量$E_{30\mathrm{min}}$\,[kWh]であるため,最適化計算の入力とするにあたり,以下の関係式を用いて平均電力$P_{\mathrm{avg}}$\,[kW]に換算した.
\begin{equation}
P_{\mathrm{avg}} = \frac{E_{30\mathrm{min}}}{0.5}
\label{eq:power_conversion}
\end{equation}

なお,目的関数における電気料金の算出に際しては,決定変数(電力\,[kW])に時間刻み$\Delta t = 0.5$\,[h]を乗じることで,再び電力量\,[kWh]ベースに換算して評価を行っている.

\subsection{計算環境と実装}

アルゴリズムの実装にはPython 3.xを使用し,混合整数計画問題(MILP)のソルバーにはSCIPのPythonインターフェースであるPySCIPOptを採用した.数値計算およびデータ処理にはpandas,numpyライブラリを,結果の可視化にはmatplotlibをそれぞれ用いた.

\subsection{最適化パラメータ}

ローリング計画法における予測ホライズン$H$は基準として96ステップ(48時間)に設定し,1ステップ(30分)ごとに計画を更新するスライディングウィンドウ方式にて最適化計算を実行した.スライディングウィンドウ方式とは,固定長の時間窓(ウィンドウ)を1ステップずつずらしながら繰り返し最適化を行う手法である.また,予測期間の影響を評価するため,$H=48$(24時間),$H=96$(48時間),$H=120$(60時間),$H=144$(72時間)の4ケースについても比較検証を行った.

\section{結果}

本章では,本研究の手法を用いたシミュレーション結果について述べる.まず,システムの基本的な運用パターンを代表的な日のデータを用いて示した後,蓄電池容量と料金プランの関係性の分析,ローリング計画法の予測期間の影響評価,およびローリング計画法と年間一括最適化の比較,の3つの観点から検証結果を述べる.

\subsection{システムの基本的な運用パターン}

需給条件の異なる代表的な日における運用挙動を図\ref{fig:battery_operation}および図\ref{fig:battery_operation_low}に示す.これらの図は蓄電池容量860kWh,北海道電力基本プランの条件下での結果である.

\noindent
\textbf{高需要日}:6月2日(需要2,436kWh,PV発電1,433kWh)と6月24日(需要2,461kWh,PV発電237kWh)を比較する.PV発電が豊富な6月2日は,昼間の余剰電力を蓄電し夜間に放電することで,買電量を1,275kWh(需要の52.4\%)に抑制した.一方,PV発電が少ない6月24日は買電量が2,080kWh(需要の84.5\%)に増加し,買電依存度の差は1.6倍となった.

\noindent
\textbf{低需要日}:2月5日(需要1,284kWh,PV発電1,019kWh)と1月22日(需要1,298kWh,PV発電281kWh)を比較する.PV発電が豊富な2月5日は買電量が270kWh(需要の21.0\%)まで減少し,高い自給率を実現した.一方,PV発電が少ない1月22日は買電量が862kWh(需要の66.4\%)となり,買電依存度の差は3.2倍に達した.この結果から,低需要期においてはPV発電量の多い・少ないが買電量に与える影響が高需要期よりも顕著であることが確認された.

\begin{figure}[H]
\centering
\includegraphics[width=\textwidth]{../png/soc860/daily_battery_pattern.png}
\caption{需要が高い日の運用パターン(需要約2,450 kWh).上:PV発電量が多い日(2024年6月2日,PV発電1,433 kWh),下:PV発電量が少ない日(2024年6月24日,PV発電237 kWh).}
\label{fig:battery_operation}
\end{figure}

\begin{figure}[H]
\centering
\includegraphics[width=\textwidth]{../png/soc860/daily_battery_pattern_low_demand.png}
\caption{需要が低い日の運用パターン(需要約1,290 kWh).上:PV発電量が多い日(2024年2月5日,PV発電1,019 kWh),下:PV発電量が少ない日(2024年1月22日,PV発電281 kWh).}
\label{fig:battery_operation_low}
\end{figure}

また,図\ref{fig:battery_operation}や図\ref{fig:battery_operation_low}において買電電力が一定値を維持する挙動が観察されるが,これは式(\ref{eq:objective})で導入した基本料金項の影響である.契約電力の増大を回避しつつ,蓄電池の充放電で需給調整を行う最適制御が機能していることを示す.

\subsection{蓄電池容量と料金プランの経済性評価}

予測期間$H=96$(48時間)の条件下において,蓄電池容量を0kWhから1720kWhまで変動させた際の経済性および運用特性を比較した.

\subsubsection{蓄電池容量による経済性の変化}

蓄電池容量ごとの年間コスト比較を表\ref{tab:capacity_plan_comparison}および表\ref{tab:optimal_comparison}に示す.解析の結果,以下の傾向が明らかとなった.

\begin{table}[H]
\centering
\caption{蓄電池容量別の年間コスト比較(両プラン)}
\label{tab:capacity_plan_comparison}
\begin{tabular}{rrrrl}
\toprule
容量 [kWh] & 北電基本 [円] & 市場連動 [円] & 差額 [円] & 有利なプラン \\
\midrule
0 & 18,101,486 & 17,422,732 & $-678,754$ & 市場連動 \\
215 & 15,585,445 & 14,787,091 & $-798,354$ & 市場連動 \\
430 & 14,941,626 & 14,613,390 & $-328,236$ & 市場連動 \\
540 & 14,731,394 & 14,859,816 & $-128,422$ & 市場連動 \\
645 & 14,561,085 & 14,858,066 & $-296,981$ & 市場連動 \\
860 & 14,343,827 & 15,023,532 & $-679,705$ & 市場連動 \\
1290 & 14,074,455 & 15,125,066 & $-1,050,611$ & 市場連動 \\
1720 & 14,072,576 & 15,290,495 & $-1,217,919$ & 市場連動 \\
\bottomrule
\end{tabular}
\end{table}

\noindent
差額は「北電基本 $-$ 市場連動」を示す.正の値は北電基本プランが安価,負の値は市場連動プランが安価であることを意味する.

\noindent
\textbf{料金プランの経済性}:全ての蓄電池容量において市場価格連動プランが経済的に有利であることが確認された.ただし,蓄電池容量540kWh以上では両プランの差額が縮小し,容量1720kWhでは差額が約122万円まで拡大する傾向が見られた.

各料金プランにとって最適な蓄電池容量での比較を表\ref{tab:optimal_comparison}に示す.

\begin{table}[H]
\centering
\caption{各プランで最もコストが低い容量での年間コスト比較}
\label{tab:optimal_comparison}
\begin{tabular}{llrr}
\toprule
料金プラン & コスト最小容量 & 年間コスト [円] & 契約電力 [kW] \\
\midrule
市場価格連動プラン & 430 kWh & 14,613,390 & 201.40 \\
北海道電力基本プラン & 1720 kWh & 14,072,576 & 157.40 \\
\midrule
\multicolumn{2}{l}{\textbf{差額}} & \multicolumn{2}{l}{\textbf{540,814円(北電基本が安価)}} \\
\bottomrule
\end{tabular}
\end{table}

\noindent
\textbf{コスト最小容量での比較}:検討した蓄電池容量(0〜1720kWh)の範囲内で各プランのコストが最小となる容量(市場連動:430kWh,北電基本:1720kWh)同士を比較した結果,北電基本プランの方が年間約54万円安価であった.なお,本比較は離散的な容量設定に基づくものであり,連続的な容量最適化を行った場合には異なる結果となる可能性がある.

\subsubsection{契約電力と運用特性の違い}

両プランの蓄電池容量別詳細比較を表\ref{tab:capacity_comparison_detail}および表\ref{tab:capacity_comparison_market_detail}に示す.なお,表中の「PV利用率」は「実際に使用したPV発電量 $\div$ PV発電可能量 $\times$ 100 [\%]」で定義され,蓄電池がない場合はPV余剰電力(PV発電量が施設の需要を上回った際に生じる余剰分)を消費できず出力抑制が発生するため100\%未満となる.料金体系の違いが契約電力(最大買電電力)に与える影響が明らかとなった.

\begin{table}[H]
\centering
\caption{蓄電池容量別の詳細比較(北海道電力基本プラン)}
\label{tab:capacity_comparison_detail}
\begin{tabular}{rrrrrr}
\toprule
容量 [kWh] & 契約電力 [kW] & 買電量 [kWh] & PV利用率 [\%] & 年間コスト [円] & コスト差 [円] \\
\midrule
0 & 262.00 & 590,587 & 78.01 & 18,101,486 & - \\
215 & 198.01 & 551,647 & 92.43 & 15,585,445 & $-2,516,041$ \\
430 & 185.50 & 535,043 & 98.41 & 14,941,626 & $-3,159,860$ \\
540 & 179.76 & 532,389 & 99.46 & 14,731,394 & $-3,370,092$ \\
645 & 174.36 & 531,536 & 99.86 & 14,561,085 & $-3,540,401$ \\
860 & 166.83 & 531,528 & 100.0 & 14,343,827 & $-3,757,659$ \\
1290 & 157.40 & 531,672 & 100.0 & 14,074,455 & $-4,027,031$ \\
1720 & 157.40 & 531,553 & 100.0 & 14,072,576 & $-4,028,910$ \\
\bottomrule
\end{tabular}
\end{table}

\begin{table}[H]
\centering
\caption{蓄電池容量別の詳細比較(市場価格連動プラン)}
\label{tab:capacity_comparison_market_detail}
\begin{tabular}{rrrrrr}
\toprule
容量 [kWh] & 契約電力 [kW] & 買電量 [kWh] & PV利用率 [\%] & 年間コスト [円] & コスト差 [円] \\
\midrule
0 & 262.00 & 590,587 & 78.01 & 17,422,732 & - \\
215 & 198.01 & 552,550 & 92.43 & 14,787,091 & $-2,635,641$ \\
430 & 201.40 & 535,623 & 98.41 & 14,613,390 & $-2,809,342$ \\
540 & 211.41 & 532,738 & 99.46 & 14,859,816 & $-2,562,916$ \\
645 & 212.00 & 531,684 & 99.86 & 14,858,066 & $-2,564,666$ \\
860 & 218.05 & 531,358 & 100.0 & 15,023,532 & $-2,399,200$ \\
1290 & 221.65 & 531,333 & 100.0 & 15,125,066 & $-2,297,666$ \\
1720 & 227.63 & 531,135 & 100.0 & 15,290,495 & $-2,132,237$ \\
\bottomrule
\end{tabular}
\end{table}

\noindent
コスト差は蓄電池容量0kWh(蓄電池なし)を基準とした差額を示す.両プランで以下の特性の違いが観察された.

図\ref{fig:capacity_contract_power}に蓄電池容量と契約電力・年間コストの関係を示す.

\begin{figure}[H]
\centering
\includegraphics[width=0.9\textwidth]{../png/soc860/capacity_contract_power.png}
\caption{蓄電池容量と契約電力・年間コストの関係}
\label{fig:capacity_contract_power}
\end{figure}
% 図が表示されていない

\noindent
\textbf{北海道電力基本プラン}:蓄電池容量の増加に伴い,契約電力は単調減少した(蓄電池なし時の262.00kWから蓄電池1720kWh時の157.40kWまで低減).これは,電力量単価が一定であるため,蓄電池が買電電力の最大値抑制に最大限活用された結果である.

\noindent
\textbf{市場価格連動プラン}:契約電力は蓄電池容量215kWhのとき198.01kWで最小となり,それ以降は容量増加に伴い逆に増大する傾向を示した(227.63kWまで上昇).これは,価格が安い時間帯に集中的な買電・充電が行われたことに起因する.

\subsubsection{年間運用統計(蓄電池容量860kWh)}

蓄電池容量860kWhにおける年間運用実績を表\ref{tab:annual_cost},表\ref{tab:system_stats}および図\ref{fig:pv_buysell},図\ref{fig:annual_soc}に示す.

\begin{table}[H]
\centering
\caption{年間電気料金の比較(蓄電池860kWh,2024年実績)}
\label{tab:annual_cost}
\begin{tabular}{lrr}
\toprule
項目 & 北海道電力基本プラン & 市場価格連動プラン \\
\midrule
基本料金 [円] & 4,814,924 & 6,293,481 \\
電力量料金 [円] & 11,433,157 & 6,615,247 \\
燃料費調整額 [円] & $-4,019,734$ & - \\
再エネ賦課金 [円] & 2,115,480 & 2,114,804 \\
\midrule
\textbf{年間合計 [円]} & \textbf{14,343,827} & \textbf{15,023,532} \\
\midrule
契約電力 [kW] & 166.83 & 218.05 \\
年間削減額 [円] & \multicolumn{2}{c}{679,705(北海道電力基本プランが安価)} \\
\bottomrule
\end{tabular}
\end{table}

\noindent
\textbf{計算条件:}北海道電力基本プランでは電力量料金21.51円/kWh+月別燃料費調整額($-5.83$〜$-9.60$円/kWh)+再エネ賦課金3.98円/kWh,市場価格連動プランではJEPX価格+再エネ賦課金3.98円/kWhを適用した.基本料金単価は$2{,}829.60 \times 0.85 \times 12$円/kW/年である.

2024年の年間システム運用統計を表\ref{tab:system_stats}に示す.なお,PV自給率は「PV発電による消費量 $\div$ 総需要 $\times$ 100 [\%]」,PV利用率は「実際に使用したPV発電量 $\div$ 発電可能量 $\times$ 100 [\%]」で定義される.

\begin{table}[H]
\centering
\caption{年間システム運用統計(2024年)}
\label{tab:system_stats}
\begin{tabular}{lrr}
\toprule
項目 & 北海道電力基本プラン & 市場価格連動プラン \\
\midrule
総需要電力量 [kWh] & 812,982 & 812,982 \\
総PV発電量 [kWh] & 287,633 & 287,633 \\
\quad PV自家消費量 [kWh] & 287,633 & 287,633 \\
\quad PV余剰量 [kWh] & 0 & 0 \\
総買電量 [kWh] & 531,528 & 531,358 \\
\midrule
PV自給率 [\%] & 35.4 & 35.4 \\
PV利用率 [\%] & 100.00 & 100.00 \\
最大買電電力 [kW] & 166.83 & 218.05 \\
平均買電電力 [kW] & 60.68 & 60.66 \\
平均SOC [kWh] & 362.83 (42.2\%) & 356.69 (41.5\%) \\
満充電回数 & 155 & 125 \\
\bottomrule
\end{tabular}
\end{table}

\noindent
年間総需要812,982kWhに対し,両プランともにPV発電量(287,633kWh)の全量を自家消費し,PV自給率35.4\%を達成した.一方で,運用パターンには差異が見られ,北電基本プランは契約電力の抑制(166.83kW)を優先して平準化された運用を行うのに対し,市場連動プランは価格変動に応じた運用を行うため,最大買電電力が218.05kWまで上昇する結果となった.
% そもそも市場連動プランの価格が、北電基本プランの基本価格とどう違うのかがすごく気になった。そこを調べないと考察が正確にできないと思う

年間のPV発電量・買電量・需要の推移を図\ref{fig:pv_buysell}に,年間の蓄電池SOC推移を図\ref{fig:annual_soc}に示す.

\begin{figure}[H]
\centering
\includegraphics[width=\textwidth]{../png/soc860/annual_pv_buy_demand.png}
\caption{年間のPV発電量・買電量・需要の推移(北海道電力基本プラン,蓄電池860kWh)}
\label{fig:pv_buysell}
\end{figure}

\begin{figure}[H]
\centering
\includegraphics[width=\textwidth]{../png/soc860/annual_soc.png}
\caption{年間の蓄電池SOC推移(北海道電力基本プラン,蓄電池860kWh,2024年)}
\label{fig:annual_soc}
\end{figure}

\subsection{予測期間の影響分析}

予測期間(ホライズン)が運用結果に与える影響について,蓄電池860kWhを対象に,24時間,48時間,60時間,72時間の4ケースで比較検証を行った.

\subsubsection{予測期間と経済効果}

各予測期間における年間統計を表\ref{tab:horizon_comparison}(市場価格連動プラン)および表\ref{tab:horizon_comparison_hokkaido}(北海道電力基本プラン)に示す.

\begin{table}[H]
\centering
\caption{予測期間による年間統計の比較(市場価格連動プラン,蓄電池860kWh)}
\label{tab:horizon_comparison}
\begin{tabular}{lrrrr}
\toprule
項目 & 24時間予測 & 48時間予測 & 60時間予測 & 72時間予測 \\
\midrule
契約電力 [kW] & 230.49 & 218.05 & 218.05 & 213.19 \\
年間買電量 [kWh] & 530,742 & 531,358 & 531,418 & 531,373 \\
年間コスト [円] & 15,404,698 & 15,023,532 & 15,003,341 & 14,847,956 \\
24時間比コスト差 [円] & - & $-$381,166 & $-$401,357 & $-$556,742 \\
\bottomrule
\end{tabular}
\end{table}

\begin{table}[H]
\centering
\caption{予測期間による年間統計の比較(北海道電力基本プラン,蓄電池860kWh)}
\label{tab:horizon_comparison_hokkaido}
\begin{tabular}{lrrrr}
\toprule
項目 & 24時間予測 & 48時間予測 & 60時間予測 & 72時間予測 \\
\midrule
契約電力 [kW] & 169.68 & 166.83 & 166.83 & 166.83 \\
年間買電量 [kWh] & 531,707 & 531,528 & 531,291 & 531,145 \\
年間コスト [円] & 14,429,874 & 14,343,827 & 14,339,597 & 14,337,000 \\
24時間比コスト差 [円] & - & $-$86,047 & $-$90,277 & $-$92,874 \\
\bottomrule
\end{tabular}
\end{table}
\noindent
市場価格連動プランでは,予測期間を24時間から48時間に延長することで年間コストが約38.1万円(2.5\%)削減された.60時間予測では48時間予測からわずか約2.0万円の追加削減,72時間予測では48時間予測から約17.6万円の追加削減となり,\textbf{改善効果の逓減}が確認された.

北海道電力基本プランでは,予測期間延長による効果がより限定的であり,24時間から72時間への延長で約9.3万円(0.6\%)の削減に留まった.これは,電力量単価が一定のため価格変動への追従が不要であり,主に買電電力の最大値抑制の精度向上のみが効果として現れるためである.

\subsubsection{計算コストとのトレードオフ}

予測期間ごとの計算時間を表\ref{tab:horizon_timing}に示す.計算環境はMacBook Pro(Apple M1 Pro, 16GB RAM)である.

\begin{table}[H]
\centering
\caption{予測期間ごとの計算時間(蓄電池860kWh)}
\label{tab:horizon_timing}
\begin{tabular}{lrrr}
\toprule
予測期間 & 北海道電力プラン [分] & 市場連動プラン [分] & 合計 [分] \\
\midrule
$H=48$(24時間) & 7.9 & 31.0 & 39.3 \\
$H=96$(48時間) & 16.1 & 16.4 & 33.2 \\
$H=120$(60時間) & 22.7 & 23.2 & 46.3 \\
$H=144$(72時間) & 32.3 & 27.9 & 60.6 \\
\bottomrule
\end{tabular}
\end{table}

\noindent
計算時間は予測期間の延長に伴い増加した.市場価格連動プランでは,48時間予測は24時間予測と比較して計算時間が短縮される一方,コストは約38.1万円削減され,費用対効果が高い.60時間予測は48時間予測から計算時間が約13.1分増加するが,追加のコスト削減はわずか約2.0万円に留まった.本システム構成と検証した4つの予測期間においては,計算資源と削減効果のバランスから\textbf{48時間(96ステップ)の予測期間が実用的な選択肢の一つ}であると考えられる.

\subsection{年間一括最適化との比較}

本節では,ローリング計画法の理論的限界を評価するため,年間全データ(17,520ステップ)を一括で最適化するシミュレーションを実施し,その結果をローリング計画法と比較する.年間一括最適化は,将来の需要・PV発電・市場価格を完全に予見した上での理論的最適解であり,ローリング計画法による解がどの程度最適に近いかを評価するベンチマークとなる.

\subsubsection{年間一括最適化の実装}

年間一括最適化では,ローリング計画法と同一の目的関数・制約条件を用いつつ,予測期間$H$を年間全ステップ($H=17{,}520$)に設定した.ただし,目的関数における基本料金項の重み係数$w_{\mathrm{basic}}$は年間按分ではなく,年間の最大買電電力が直接契約電力となるよう式(\ref{eq:weight_basic})において$H = 17{,}520$として計算した.計算時間を1時間に制限し,ソルバー(SCIP)が最適解またはそれに準ずる解を発見した時点で終了するよう設定した.

\subsubsection{蓄電池容量別の比較結果}

蓄電池容量430kWh,860kWh,1290kWhの3ケースについて,ローリング計画法(48時間予測)と年間一括最適化の結果を比較した.表\ref{tab:annual_vs_rolling_hokkaido}に北海道電力基本プラン,表\ref{tab:annual_vs_rolling_market}に市場価格連動プランの結果を示す.

\begin{table}[H]
\centering
\caption{ローリング計画法と年間一括最適化の比較(北海道電力基本プラン)}
\label{tab:annual_vs_rolling_hokkaido}
\begin{tabular}{lrrrrr}
\toprule
容量 [kWh] & \multicolumn{2}{c}{年間コスト [万円]} & コスト差 & \multicolumn{2}{c}{契約電力 [kW]} \\
\cmidrule(lr){2-3} \cmidrule(lr){5-6}
 & ローリング & 年間一括 & [万円] (\%) & ローリング & 年間一括 \\
\midrule
430 & 1,494.2 & 1,490.0 & $-$4.2 (0.28\%) & 185.5 & 185.5 \\
860 & 1,434.4 & 1,427.8 & $-$6.6 (0.46\%) & 166.8 & 166.8 \\
1290 & 1,407.4 & 1,397.8 & $-$9.6 (0.69\%) & 157.4 & 156.4 \\
\bottomrule
\end{tabular}
\end{table}

\begin{table}[H]
\centering
\caption{ローリング計画法と年間一括最適化の比較(市場価格連動プラン)}
\label{tab:annual_vs_rolling_market}
\begin{tabular}{lrrrrr}
\toprule
容量 [kWh] & \multicolumn{2}{c}{年間コスト [万円]} & コスト差 & \multicolumn{2}{c}{契約電力 [kW]} \\
\cmidrule(lr){2-3} \cmidrule(lr){5-6}
 & ローリング & 年間一括 & [万円] (\%) & ローリング & 年間一括 \\
\midrule
430 & 1,461.3 & 1,338.9 & $-$122.4 (8.4\%) & 201.4 & 185.5 \\
860 & 1,502.4 & 1,262.9 & $-$239.5 (15.9\%) & 218.1 & 166.8 \\
1290 & 1,512.5 & 1,233.4 & $-$279.1 (18.5\%) & 221.6 & 156.4 \\
\bottomrule
\end{tabular}
\end{table}

\noindent
北海道電力基本プランでは,ローリング計画法と年間一括最適化のコスト差は0.28\%〜0.69\%と小さく,ローリング計画法でもほぼ最適解に近い結果が得られていることが確認された.一方,市場価格連動プランでは8.4\%〜18.5\%と大きな差が生じた.この差異の原因について次節で考察する.

\subsubsection{計算時間の比較}

年間一括最適化の計算時間を表\ref{tab:annual_timing}に示す.

\begin{table}[H]
\centering
\caption{年間一括最適化の計算時間}
\label{tab:annual_timing}
\begin{tabular}{lrr}
\toprule
容量 [kWh] & 北海道電力プラン [分] & 市場連動プラン [分] \\
\midrule
430 & 2.2 & 1.9 \\
860 & 2.0 & 2.5 \\
1290 & 2.6 & 2.2 \\
\bottomrule
\end{tabular}
\end{table}

\noindent
年間一括最適化は約2分で完了し,ローリング計画法(約35〜40分)の約\textbf{20分の1}の計算時間であった.これは,ローリング計画法が17,520回の小規模最適化問題を逐次解くのに対し,年間一括最適化は1回の大規模最適化問題を解くためである.問題規模は大きいが(約87,000変数,87,000制約),現代のMIPソルバー(SCIP)の効率的な前処理とカット生成により,短時間で最適解が発見された.

\subsubsection{料金プラン優位性の再評価}

年間一括最適化の結果に基づき,料金プランの優位性を再評価した(表\ref{tab:plan_comparison_annual}).

\begin{table}[H]
\centering
\caption{年間一括最適化による料金プラン比較}
\label{tab:plan_comparison_annual}
\begin{tabular}{lrrrl}
\toprule
容量 [kWh] & 北電基本 [万円] & 市場連動 [万円] & 差額 [万円] & 有利なプラン \\
\midrule
430 & 1,490.0 & 1,338.9 & $-$151.1 & \textbf{市場連動} \\
860 & 1,427.8 & 1,262.9 & $-$164.9 & \textbf{市場連動} \\
1290 & 1,397.8 & 1,233.4 & $-$164.4 & \textbf{市場連動} \\
\bottomrule
\end{tabular}
\end{table}

\noindent
ローリング計画法では蓄電池容量540kWh以上で北海道電力基本プランが有利となる「優位性逆転」が観察された(表\ref{tab:capacity_plan_comparison})が,年間一括最適化では\textbf{全ての容量において市場価格連動プランが有利}という結果となった.この差異は,ローリング計画法における市場価格連動プランの運用が局所最適に陥っていることを示唆する.

\section{考察}

\subsection{料金プランと蓄電池容量の経済性評価}

本節では,シナリオAの結果に基づき,蓄電池容量がある程度大きい場合に北海道電力基本プランが市場価格連動プランより有利となる理由を分析する.

\subsubsection{料金プランの優位性が逆転するメカニズム}

蓄電池容量430kWh以下では市場価格連動プランが有利,540kWh以上では北海道電力基本プランが有利となる逆転現象が確認された(表\ref{tab:capacity_plan_comparison}).この逆転は,両プランにおける蓄電池の最適運用戦略の違いに起因すると考えられる.

\

\textbf{北海道電力基本プラン}:電力量単価が一定(21.51円/kWh)であるため,買電のタイミングによる電力量料金の差は生じない.このため,最適化において電力量料金を削減する動機がなく,基本料金(契約電力)の削減が唯一の最適化目標となる.結果として,蓄電池は買電電力の最大値抑制に最大限活用され,蓄電池容量の増加に伴い契約電力は単調減少し(262.00kW→157.40kW),コスト削減効果が継続する.

\

\textbf{市場価格連動プラン}:JEPX価格が時間帯により変動する(3.80〜31.00円/kWh)ため,安価な時間帯に買電を集中させる時間的裁定が有効となる.しかし,この戦略は買電の最大電力を増大させ,契約電力の上昇を招く.蓄電池容量430kWh以降では,契約電力増加による基本料金の増加が時間的裁定による電力量料金の削減を上回り,コストが増加に転じる(契約電力:201.40kW→227.63kW).

\subsubsection{蓄電池導入による経済効果}

蓄電池導入効果を表\ref{tab:battery_effect}に示す.

\begin{table}[H]
\centering
\caption{蓄電池導入効果の比較(蓄電池0kWh vs 860kWh)}
\label{tab:battery_effect}
\begin{tabular}{llrrr}
\toprule
項目 & 料金プラン & 蓄電池なし & 蓄電池860kWh & 削減額/削減率 \\
\midrule
\multirow{2}{*}{年間コスト [万円]} & 北海道電力 & 1,810.1 & 1,434.4 & 375.7 (20.8\%) \\
 & 市場連動 & 1,742.3 & 1,502.4 & 239.9 (13.8\%) \\
\midrule
\multirow{2}{*}{契約電力 [kW]} & 北海道電力 & 262.0 & 166.8 & 95.2 (36.3\%) \\
 & 市場連動 & 262.0 & 218.1 & 43.9 (16.8\%) \\
\midrule
\multirow{2}{*}{年間買電量 [MWh]} & 北海道電力 & 590.6 & 531.5 & 59.1 (10.0\%) \\
 & 市場連動 & 590.6 & 531.4 & 59.2 (10.0\%) \\
\midrule
PV利用率 [\%] & 両プラン共通 & 78.0 & 100.0 & \\
\bottomrule
\end{tabular}
\end{table}


\noindent
北海道電力基本プランでは年間\textbf{375.7万円(20.8\%)},市場価格連動プランでは年間\textbf{239.9万円(13.8\%)}のコスト削減を達成した.北海道電力基本プランの削減効果がより大きい理由は,契約電力の削減率(36.3\% vs 16.8\%)の差に起因する.

蓄電池なしでは市場価格連動プランが有利(年間67.9万円の差)であったが,蓄電池860kWhでは北海道電力基本プランが有利(年間68.0万円/年)となり,\textbf{蓄電池導入により最適な料金プランが逆転した}.

\subsubsection{最適プランの結論と留意事項}

各プランの最適容量(市場連動:430kWh,北電基本:1720kWh)で比較した場合でも,北海道電力基本プランの方が年間約54万円安価である(表\ref{tab:optimal_comparison}).蓄電池860kWhと北海道電力基本プランの組み合わせは,蓄電池なし・市場価格連動プランと比較して年間\textbf{307.9万円}のコスト削減が可能である.

\noindent
\textbf{留意事項}:上記の比較は年間運用コスト(OPEX)のみに基づいており,蓄電池の初期投資コスト(CAPEX)は考慮していない.最適蓄電池容量が430kWhと1720kWhで約4倍異なるため,実際の投資判断においては蓄電池単価,耐用年数,割引率等を考慮したライフサイクルコスト分析が必要である.

\subsection{季節変動と市場価格特性による要因分析}

本節では,北海道電力基本プランが有利となる背景を,季節別の蓄電池効果と市場価格特性の両面から分析する.

\subsubsection{季節別の蓄電池効果}

蓄電池860kWhの運用データを月別に分析した結果を表\ref{tab:monthly_analysis}に示す.ここで,買電抑制率は$(\text{需要の最大電力} - \text{買電の最大電力}) / \text{需要の最大電力} \times 100$で算出した値であり,蓄電池によって買電の最大電力がどの程度抑制されたかを示す.

\begin{table}[H]
\centering
\caption{月別エネルギー統計と蓄電池効果}
\label{tab:monthly_analysis}
\begin{tabular}{lrrrrrr}
\toprule
月 & 需要 [MWh] & PV [MWh] & 買電 [MWh] & 需要最大 [kW] & 買電最大 [kW] & 買電抑制率 [\%] \\
\midrule
1月 & 40.2 & 20.9 & 19.7 & 130.0 & 41.0 & 68.5 \\
2月 & 36.7 & 25.3 & 12.3 & 150.0 & 29.0 & 80.7 \\
3月 & 49.0 & 32.4 & 16.8 & 154.0 & 36.2 & 76.5 \\
4月 & 54.9 & 26.6 & 29.0 & 220.0 & 64.0 & 70.9 \\
5月 & 66.6 & 27.1 & 39.9 & 204.0 & 80.4 & 60.6 \\
6月 & 77.6 & 24.8 & 53.3 & 246.0 & 121.8 & 50.5 \\
7月 & 111.2 & 25.6 & 86.2 & 260.0 & 154.9 & 40.4 \\
8月 & 118.2 & 18.8 & 99.8 & 264.0 & 166.8 & 36.8 \\
9月 & 90.9 & 24.9 & 66.7 & 242.0 & 139.0 & 42.6 \\
10月 & 72.5 & 22.0 & 50.9 & 218.0 & 115.5 & 47.0 \\
11月 & 49.9 & 19.5 & 30.8 & 144.0 & 61.0 & 57.6 \\
12月 & 45.3 & 19.8 & 25.8 & 150.0 & 43.5 & 71.0 \\
\midrule
年間 & 813.0 & 287.6 & 531.4 & 264.0 & 166.8 & -- \\
\bottomrule
\end{tabular}
\end{table}

\noindent
月別分析から,以下の知見が得られた:

\begin{enumerate}
    \item \textbf{冬季(12--2月)の蓄電池効果が最大}:買電抑制率68.5〜80.7\%と高い値を達成した.これは,冬季の需要122.2MWhに対しPV発電65.9MWh(需要の54\%)が確保されており,蓄電池容量860kWhが需要に対して相対的に大きいためである.特に2月は買電抑制率80.7\%と最高値を記録した.

    \item \textbf{夏季(6--8月)の蓄電池効果が最小}:買電抑制率36.8〜50.5\%に留まった.夏季は需要307.0MWhと年間最大である一方,PV発電は69.3MWh(需要の23\%)に過ぎない.8月は需要118.2MWhに対しPV発電18.8MWhと最も需給バランスが悪く,蓄電池容量860kWhでは買電の最大電力を十分に抑制できない.

    \item \textbf{PV発電量の季節パターン}:3月が32.4MWhで最大,8月が18.8MWhで最小となった.

    \item \textbf{契約電力への示唆}:年間の買電最大電力は8月に発生(166.8kW)した.夏季は買電抑制率が低く(項目2参照),かつ需要の最大電力が大きい(8月:264.0kW)ため,抑制後の買電最大電力も高くなる.

    \item \textbf{蓄電池サイクル数}:年間推定218サイクル(月平均18サイクル)であり,一般的なリチウムイオン電池の寿命(3,000〜6,000サイクル)に対して十分な余裕がある.なお,1サイクルとは蓄電池容量相当の電力量を充放電する単位を指し,部分的な充放電は累積して換算する.
\end{enumerate}

図\ref{fig:monthly_analysis}に月別エネルギー量と買電抑制率のグラフを示す.

\begin{figure}[H]
    \centering
    \includegraphics[width=0.95\textwidth]{../png/soc860/monthly_analysis.png}
    \caption{月別エネルギー量と蓄電池効果の分析.左下グラフの「PV自家消費率」は需要に対するPV発電の直接消費割合,「蓄電池放電貢献率」は需要に対する蓄電池放電量の割合を示す.右下グラフの「買電抑制率」は $(\text{需要の最大電力} - \text{買電の最大電力}) / \text{需要の最大電力} \times 100$ で算出.}
    \label{fig:monthly_analysis}
\end{figure}

\subsubsection{市場価格(JEPX)の季節変動特性}

JEPXスポット価格の月別統計を表\ref{tab:price_monthly}に示す.ここで,「$>$北電率」は市場価格が北海道電力基本プランの電力量料金(21.51円/kWh)を上回る時間帯の割合を示す.

\begin{table}[H]
\centering
\caption{月別市場価格統計}
\label{tab:price_monthly}
\begin{tabular}{lrrrrrr}
\toprule
月 & 平均 [円/kWh] & 中央値 [円/kWh] & 最小 [円/kWh] & 最大 [円/kWh] & $>$北電率 [\%] & 高価格回数 \\
\midrule
1月 & 14.07 & 13.79 & 3.99 & 23.10 & 0.4 & 0 \\
2月 & 13.42 & 13.00 & 3.99 & 25.22 & 1.9 & 2 \\
3月 & 15.94 & 15.79 & 3.99 & 40.57 & 11.4 & 16 \\
4月 & 14.30 & 15.52 & 3.99 & 23.98 & 4.4 & 0 \\
5月 & 15.67 & 16.03 & 3.99 & 25.88 & 14.0 & 4 \\
6月 & 16.72 & 16.52 & 3.99 & 25.75 & 15.4 & 12 \\
7月 & 18.39 & 17.48 & 3.99 & 33.98 & 26.1 & 124 \\
8月 & 19.15 & 19.62 & 8.99 & 25.31 & 32.8 & 60 \\
9月 & 19.27 & 17.98 & 3.99 & 53.63 & 30.3 & 104 \\
10月 & 18.75 & 18.05 & 3.99 & 42.98 & 37.6 & 30 \\
11月 & 18.78 & 18.45 & 6.98 & 25.30 & 29.4 & 30 \\
12月 & 17.70 & 18.09 & 4.00 & 26.98 & 16.4 & 18 \\
\bottomrule
\end{tabular}
\end{table}

\noindent
「高価格回数」は25円/kWh以上の価格が発生した30分コマ数を示す.年間最高価格53.63円/kWhは2024年9月20日9:00に発生した.

市場価格分析から,以下の知見が得られた:

\begin{enumerate}
    \item \textbf{市場価格の年間平均は北海道電力より低い}:年間平均16.87円/kWhは北海道電力基本プラン(21.51円/kWh)より\textbf{21.6\%安い}.しかし,この単純比較は買電タイミングを考慮していない.

    \item \textbf{市場価格が北海道電力を上回る時間帯は18.5\%}:年間17,520コマのうち3,236コマで市場価格が21.51円/kWhを超過する.特に秋季(9--11月)は32.5\%と最も高く,冬季(12--2月)は6.4\%と最も低い.

    \item \textbf{高価格の季節集中}:25円/kWh以上の高価格は年間400回(2.3\%)発生し,7月(124回),9月(104回),8月(60回)に集中している.これは冷房需要増加による電力逼迫を反映している.

    \item \textbf{高価格の時間帯集中}:25円/kWh以上の高価格は9:00--14:00に集中しており(上位5時間帯で全体の57.5\%),昼間の需要が高い時間帯と一致する.この時間帯は需要も高いため,市場価格連動プランでは高価格での買電が避けられない.

    \item \textbf{時間帯別価格パターン}:深夜(0:00--2:00)は平均11--12円/kWh,昼間(11:00--14:00)は平均20--21円/kWhと,約2倍の価格差がある.市場価格連動プランはこの価格差を利用した時間シフトが可能だが,需要パターンとの制約により完全な活用は困難である.
\end{enumerate}

\subsubsection{北海道電力基本プランが有利となる理由}

以上の分析から,北海道電力基本プランが市場価格連動プランより年間約64万円安価となる理由は以下のように説明できる:

\begin{enumerate}
    \item \textbf{高価格帯の回避の困難さ}:市場価格連動プランでは,需要が高い時間帯(昼間)に高価格が発生するため,高価格での買電が避けられない.蓄電池による時間シフトを行っても,夏季・秋季の需要が蓄電池容量(860kWh)を大きく上回るため,高価格帯の回避には限界がある.

    \item \textbf{契約電力の増加}:市場価格連動プランでは安価な時間帯に集中的に買電・充電するため,買電の最大電力が上昇し契約電力が増加する(北海道電力170.93kW vs 市場連動221.56kW).基本料金の差額(約190万円/年)が電力量料金の差額を相殺する.

    \item \textbf{季節変動との相性}:冬季(12--2月)は市場価格が最も低く(平均15.12円/kWh),かつ蓄電池効果が最も高い(買電抑制率64.3\%)季節である.しかし,この時期は需要も低いため,市場価格連動プランのメリットが限定的となる.夏季・秋季は市場価格が高く,かつ蓄電池効果が低いため,市場価格連動プランに不利に働く.
\end{enumerate}

図\ref{fig:price_seasonal_analysis}に市場価格の季節変動と北海道電力基本プランとの比較を示す.

\begin{figure}[H]
    \centering
    \includegraphics[width=0.95\textwidth]{../png/soc860/price_seasonal_analysis.png}
    \caption{市場価格(JEPX)の季節変動分析}
    \label{fig:price_seasonal_analysis}
\end{figure}


\subsection{予測期間と計算コストの妥当性}

シナリオB(予測期間比較)の結果に基づき,予測期間の選択と計算コストのトレードオフを考察する.

市場価格連動プランでは,予測期間を24時間から48時間に延長することで年間コストが約47.6万円(3.0\%)削減された.しかし,48時間から60時間への延長では追加削減がわずか約2.1万円,72時間への延長でも約16.9万円に留まり,\textbf{改善効果の逓減}が確認された(表\ref{tab:horizon_comparison}).

一方,北海道電力基本プランでは予測期間延長の効果がより限定的であり,24時間から72時間への延長で約9.1万円(0.6\%)の削減に留まった(表\ref{tab:horizon_comparison_hokkaido}).これは,電力量単価が一定であるため価格変動への追従が不要であり,買電電力の最大値抑制の精度向上のみが効果として現れるためである.

計算時間は予測期間の延長に伴いほぼ線形に増加した(24時間:12.8分 → 48時間:35.1分 → 60時間:43.7分 → 72時間:58.9分,表\ref{tab:horizon_timing}).費用対効果の観点から,市場価格連動プランでは48時間予測が最も効率的であり,60時間以上への延長は計算コストに見合わない.本システム構成と検証した4つの予測期間(24/48/60/72時間)においては,計算資源と削減効果のバランスから\textbf{48時間(96ステップ)の予測期間が実用的な選択肢の一つ}であると考えられる.

\subsection{年間一括最適化によるローリング計画法の評価}

年間一括最適化との比較の結果から,ローリング計画法の最適性について以下の知見が得られた.

\subsubsection{プラン別の最適性評価}

北海道電力基本プランでは,ローリング計画法と年間一括最適化のコスト差は0.28\%〜0.69\%と極めて小さく,ローリング計画法が\textbf{ほぼ最適解}を達成していることが確認された.これは,電力量料金が一定であるため,ソルバーが各予測期間で買電電力の平準化を行えば,自動的に年間全体での最適運用に近づくためと考えられる.

一方,市場価格連動プランでは8.4\%〜18.5\%という大きなコスト差が生じた.この差異は以下の原因があると考えられる:

\begin{enumerate}
    \item \textbf{時間的裁定の局所最適化}:ローリング計画法では,各48時間の窓内で「安価な時間帯に買電を集中」という戦略が最適となる.しかし,この局所最適が年間を通して繰り返されると,特定の時間帯(深夜帯など)に買電が集中し,契約電力が増大する.

    \item \textbf{契約電力の増大}:表\ref{tab:annual_vs_rolling_market}に示すように,ローリング計画法では契約電力が201.4〜221.6kWまで増大するのに対し,年間一括最適化では156.4〜185.5kWに抑制される.この差(最大65kW)により,基本料金が約190万円/年増加する.

    \item \textbf{蓄電池容量との関係}:蓄電池容量が大きいほどコスト差率が大きくなる(430kWh: 8.4\%,1290kWh: 18.5\%)傾向は,大容量蓄電池があるほど時間的裁定の余地が大きく,その局所最適化による弊害も大きくなることを示唆する.
\end{enumerate}

\subsubsection{料金プラン優位性への示唆}

本研究の主要な発見の一つである「蓄電池容量540kWh以上で北海道電力基本プランが有利」という結論(表\ref{tab:capacity_plan_comparison})は,ローリング計画法を用いた場合の結果である.年間一括最適化では全容量において市場価格連動プランが有利となった(表\ref{tab:plan_comparison_annual}).

この差異は,ローリング計画法の局所最適化が市場価格連動プランに不利に働くことを示している.言い換えれば,現実の運用においても市場価格変動を「48時間先まで」しか見通せない場合,大容量蓄電池を持つ施設は北海道電力基本プランの方が安全な選択となる可能性がある.一方,より長期の価格予測精度が向上すれば,市場価格連動プランの優位性を活かせる可能性がある.

\subsection{モデルの制約と限界}
\label{sec:wbasic_limitation}

本節では,採用した最適化手法の限界と,それが結果に与える影響を明確にする.

\subsubsection{基本料金係数の按分手法}

契約電力(基本料金の決定要因)は,本来「年間17,520ステップの中の最大買電電力」という\textbf{大域的な指標}で決定される.しかし,ローリング計画法では年間を通した最適化が計算上困難であるため,各予測期間(96ステップ,48時間)内の最大値$s^{\mathrm{BY}}_{\mathrm{MAX}}$に按分係数$w_{\mathrm{basic}}$を乗じてペナルティを与える近似を採用した.

この「局所的な最大値の抑制」を通じて「大域的な最大値」を間接的に制御する手法には,以下の数理的問題がある:

\begin{enumerate}
    \item \textbf{最適化基準と評価基準の乖離}:ソルバーは各予測期間で「$w_{\mathrm{basic}} \times s^{\mathrm{BY}}_{\mathrm{MAX}}$」を最小化しようとするが,最終的な基本料金は「年間最大買電電力 × 単価」で計算される.

    \item \textbf{局所最適と大域最適の不一致}:ある予測期間で買電電力の最大値を抑制する努力が,別の期間で発生するより大きな買電電力により無効化される可能性がある.

    \item \textbf{按分係数の重みの影響}:$w_{\mathrm{basic}}$の値によって,ソルバーの「買電電力の最大値抑制」と「電力量料金削減」のトレードオフ判断が変化する.
\end{enumerate}

北海道電力基本プランでは電力量料金が一定であるため,ソルバーは自然に買電電力を平準化する方向に最適化を行い,按分手法の影響は限定的である.一方,市場価格連動プランではJEPX価格の変動(3.80〜31.00円/kWh)により,「安価な時間帯への買電集中」と「買電電力の最大値抑制」のトレードオフが生じ,大容量蓄電池ほど契約電力が増大する傾向が確認された.

\subsubsection{結論の解釈における注意点}

本研究の結論を解釈する際には以下の点に注意が必要である:

\begin{enumerate}
    \item 市場価格連動プランの「最適容量430kWh」は按分係数の設定値に依存する可能性がある
    \item 契約電力の増加現象は,時間的裁定の本質的特性と按分手法の限界の両方が寄与している
    \item 両プラン比較において,按分手法の影響に非対称性がある
\end{enumerate}

\noindent
\textbf{今後の課題}:按分手法の限界を克服するためには,$w_{\mathrm{basic}}$の感度分析,年間契約電力の明示的追跡機構の導入,異なる最適化手法との比較検証が必要である.

\subsection{一般化可能性と適用範囲}

本研究の結論は,以下の特定条件に依存しており,異なる条件下では結論が変化する可能性がある:

\begin{itemize}
    \item \textbf{対象期間}:2024年1月〜12月のJEPX北海道エリア価格(年間平均16.87円/kWh)
    \item \textbf{地域}:北海道エリア(冬季暖房需要が顕著,本州との連系線容量が限定的)
    \item \textbf{需要特性}:年間812,982kWh,最大電力264.0kW,PV自給率35.4\%
    \item \textbf{システム構成}:PV 250kW,蓄電池出力400kW,売電不可
\end{itemize}

他の電力エリア,異なる年度の市場価格,異なる需要パターン(オフィスビル,工場等),または売電可能なシステム構成では,最適な料金プランや蓄電池容量が変化する.

ただし,以下の\textbf{定性的知見}は一定の普遍性を持つ:

\begin{enumerate}
    \item 料金プランの有利性は蓄電池容量に依存し,逆転する分岐点が存在しうる
    \item 固定価格プランでは契約電力の抑制,変動価格プランでは時間的裁定が最適化の主軸となる
    \item 蓄電池効果は季節により大きく変動する
    \item 予測期間の延長効果は逓減する
\end{enumerate}

異なる条件下での最適解を求めるには,本研究で開発した最適化フレームワークに対象地域の市場価格データ,料金体系,需要パターンを入力して再計算する必要がある.

\section{おわりに}

本研究では,北海道十勝地方のPV・蓄電池システム(PV容量250kW)を対象に,ローリング計画法による運用最適化シミュレーションを行い,料金プランと蓄電池容量の関係性について検証した.対象期間は2024年1月1日から12月31日(2月29日を除く365日,17,520ステップ)である.主な結論は以下の通りである.

\begin{enumerate}
    \item \textbf{料金プランの優位性と蓄電池容量}:料金プランの経済的優位性は蓄電池容量に依存することが明らかになった.容量430kWh以下では市場価格連動プランが有利である一方,540kWh以上では北海道電力基本プランが有利となり,両プランの優劣が逆転する分岐点が存在する.

    \item \textbf{最適容量における比較}:各プランのコスト最小化を実現する最適容量(市場価格連動プラン:430kWh,北海道電力基本プラン:1720kWh)において比較を行った結果,北海道電力基本プランの方が年間\textbf{約54万円安価}であり,経済的に有利であるとの結論を得た.

    \item \textbf{契約電力への影響}:北海道電力基本プランでは蓄電池容量の増加に伴い契約電力が減少するのに対し,市場価格連動プランでは安価な時間帯への買電集中により,一定容量(430kWh)を超えると契約電力が増加する傾向が確認された.

    \item \textbf{蓄電池導入効果}:蓄電池なし(0kWh)と比較して,北海道電力基本プラン+1720kWhで年間約403万円,市場価格連動プラン+430kWhで年間約281万円のコスト削減を達成した.なお,PV利用率の向上(78\%→100\%)は,蓄電池の物理的な時間シフト機能による寄与が主である.

    \item \textbf{予測期間の妥当性}:予測期間の延長はコスト削減に寄与するが,その効果は逓減する.計算コストと削減効果のトレードオフを考慮すると,本システム構成においては48時間(96ステップ)の予測期間が実用的であると結論付けられる.

    \item \textbf{年間一括最適化との比較}:ローリング計画法と年間一括最適化(パーフェクトフォーサイト)の比較により,北海道電力基本プランではローリング計画法がほぼ最適解(コスト差0.3〜0.7\%)を達成するのに対し,市場価格連動プランでは8〜18\%のコスト差が生じることが判明した.これは,ローリング計画法における時間的裁定の局所最適化が契約電力増大を招くためであり,市場価格連動プランは長期の価格予測精度が高い場合に真価を発揮する可能性がある.
\end{enumerate}

\end{document}
