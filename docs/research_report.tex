\documentclass[a4paper,12pt,dvipdfmx]{jsarticle}

\usepackage[top=20mm, bottom=25mm, left=20mm, right=20mm]{geometry}
\usepackage{url}
\usepackage{graphicx}
\usepackage{amssymb}
\usepackage{amsmath}
\usepackage{booktabs}
\usepackage{float}
\usepackage{multirow}
\usepackage{tikz}
\usetikzlibrary{shapes,arrows,positioning}

\title{電力グリッドのエネルギー需給マネジメントにおけるローリング計画モデル}
\author{神戸大学工学部情報知能工学科4年\\山崎博之}
\date{2026年2月19日}

\begin{document}

\maketitle

\section{背景と目的}

本研究では,北海道十勝地方に設置された出力250kWの太陽光発電(PV)・蓄電池システムを対象に,ローリング計画法を用いた年間電気料金の最小化について検討する.

シミュレーションには,2024年1月1日から12月31日までの365日間(2月29日を除く)における施設の電力消費量,PV発電量,および日本卸電力取引所(JEPX)のスポット価格の実測データ(30分間隔,計17,520ステップ)を用いた.料金体系として,北海道電力の基本プラン(高圧電力・固定料金)と市場価格連動プランの2種類を設定した.最適化手法には混合整数線形計画法(MILP)を採用し,ソルバーにはPySCIPOptを用いた.

本研究の目的は,2つの料金プラン(北海道電力基本プランおよび市場価格連動プラン)を対象として,ローリング計画法による運用最適化の有効性を実証することである.比較にあたり,以下の3つの軸を設定し,各条件下での経済性を体系的に評価する:
\begin{enumerate}
    \item \textbf{蓄電池容量}:0kWhから1720kWhまで変化させ,容量増加が両プランに与える影響を比較する.
    \item \textbf{予測期間}:24時間から21日間まで変化させ,予測期間の延長が最適化精度に与える影響を分析する.
    \item \textbf{再計画間隔}:30分から12時間まで変化させ,計算効率と最適化精度のバランスを検討する.
\end{enumerate}
また,ローリング計画法の理論的限界を評価するため,年間一括最適化との比較も行う.

本論文の構成は以下の通りである.第2章ではシステム構成と制約条件を述べ,第3章で最適化手法(ローリング計画法およびMILP定式化)を説明する.第4章で実験設定を示し,第5章で結果と考察を報告する.第6章で結論を述べる.

\section{システム構成と制約条件}

\subsection{用語の定義}

本論文で使用する主要な用語を以下に定義する.

\begin{itemize}
    \item \textbf{電力系統(系統)}:発電所から電力消費者(施設や家庭など)まで電力を供給するための送電・配電網の総称.本研究では,施設が電力会社から電力を購入する際の接続先を指す.
    \item \textbf{買電}:電力系統から電力を購入すること.その電力量を買電量 [kWh],瞬時電力を買電電力 [kW] と呼ぶ.
    \item \textbf{逆潮流}:施設側から電力系統へ電力を送り返すこと(売電).本研究では逆潮流を禁止する設定とした.
    \item \textbf{契約電力}:電力会社との契約で定める最大需要電力 [kW].高圧契約の場合,過去1年間の各月の最大需要電力(30分平均値)のうち最大のものが適用される.基本料金はこの契約電力に比例する.
    \item \textbf{出力抑制}:PV発電可能量の一部を意図的に使用せず捨てること.蓄電池が満充電かつ需要が少ない場合などに発生する.
\end{itemize}

\subsection{対象システムの概要}

本研究では,2024年1月1日から同年12月31日までの365日間(2月29日を除く17,520ステップ,30分時間解像度)を対象期間とし,以下の仕様を持つシステムについてシミュレーションを行った.

\begin{itemize}
    \item \textbf{太陽光発電(PV)システム}:定格出力は250kWであり,パネルは南向き,設置角度40°で固定されている.
    \item \textbf{蓄電池システム}:蓄電池容量は本研究の比較軸の1つであり,0kWhから1720kWhの範囲で変化させて経済性を評価する.最大充放電出力は400kWである.初期SOC(State of Charge:蓄電池の充電残量を0〜100\%で表す指標)は容量の50\%とし,充放電効率はいずれも0.98(リチウムイオン電池の一般的な値)と設定した.
    \item \textbf{系統電力}:買電単価は市場価格連動(JEPX)または固定料金を適用する.基本料金単価は北海道電力の高圧電力料金(2,829.60円/kW)に基づき,力率割引(85\%)を考慮した値を採用した.なお,本システムは完全自家消費型であり,逆潮流は行わない設定とした.
\end{itemize}

システム構成の概念図を図\ref{fig:system_config}に示す.

\begin{figure}[H]
\centering
\begin{tikzpicture}[
    node distance=1.5cm,
    block/.style={rectangle, draw, minimum width=2.5cm, minimum height=1cm, align=center},
    arrow/.style={->, thick},
    darrow/.style={<->, thick}
]
% ノード定義
\node[block] (grid) {電力系統\\(北電/JEPX)};
\node[block, right=2.5cm of grid] (load) {施設負荷\\(需要)};
\node[block, above=1.2cm of load] (pv) {太陽光発電\\(PV: 250kW)};
\node[block, below=1.2cm of load] (battery) {蓄電池\\(860kWh)};

% 矢印
\draw[arrow] (grid) -- node[above] {$s^{\mathrm{BY}}_k$} node[below] {買電} (load);
\draw[arrow] (pv) -- node[right] {$g^{\mathrm{P2}}_k$} (load);
\draw[darrow] (battery) -- node[right] {$x^{\mathrm{FC}}_k / x^{\mathrm{FD}}_k$} node[left] {充放電} (load);

% 注釈
\node[below=0.3cm of battery, text width=3cm, align=center, font=\small] {SOC: $b^{\mathrm{F}}_k$};
\end{tikzpicture}
\caption{システム構成の概念図.電力系統からの買電,PV発電,蓄電池の充放電により施設需要を満たす.}
\label{fig:system_config}
\end{figure}

\subsection{運用制約条件}

システム運用における最適化計算では,以下の物理的および制度的制約を課した.

\begin{itemize}
    \item \textbf{電力需給平衡}:各タイムステップにおいて,供給電力(PV発電,買電,蓄電池放電)の総和は,需要電力(施設需要,蓄電池充電)の総和と常に一致しなければならない.
    \item \textbf{蓄電池運用制約}:蓄電池の劣化抑制および安全性を考慮し,SOCの運用範囲は定格容量の5\%から95\%に制限した(容量860kWhの場合,43kWh以上817kWh以下).また,充放電電力は最大出力(400kW)以下とし,充電と放電の同時実行を禁止する排他制御も行う.
    \item \textbf{逆潮流禁止制約}:PV発電の余剰電力が生じた場合でも,売電量は常に0とし,システム内で消費または出力抑制するものとした.
\end{itemize}


\section{最適化手法}

本研究では,不確実性を伴う環境下での運用計画立案にローリング計画法(Rolling Horizon Approach)と混合整数線形計画法(MILP: Mixed Integer Linear Programming)を組み合わせて適用する.本章ではまず各手法の概要を述べ,次に本研究への適用方法を説明する.

\subsection{ローリング計画法の概要}

ローリング計画法(Rolling Horizon Approach,または Receding Horizon Control)は,長期間の最適化問題を有限の予測期間に分割して逐次的に解く手法である.本手法は以下の特徴を持つ:

\begin{itemize}
    \item \textbf{有限の予測期間}:無限または非常に長い計画期間を,有限の予測期間$H$に限定して最適化を行う.
    \item \textbf{逐次再計画}:予測期間の一部(再計画間隔$m$)のみを実行し,新たな情報を得た後に再度最適化を行う.
    \item \textbf{フィードバック}:実行結果を次回の最適化に反映することで,予測誤差や外乱に対する安定性を確保する.
\end{itemize}

この手法は,モデル予測制御(MPC: Model Predictive Control)と密接に関連しており,化学プラント制御,電力系統運用,サプライチェーン管理など幅広い分野で応用されている.計算負荷と最適性のトレードオフが存在し,予測期間$H$が長いほど大域的最適解に近づくが,計算コストが増大する.

\subsection{混合整数線形計画法(MILP)の概要}

混合整数線形計画法(MILP: Mixed Integer Linear Programming)は,線形計画法に整数制約を加えた最適化手法である.決定変数の一部が整数値(特に0-1の二値)に限定される問題を扱うことができる.

\noindent
\textbf{標準形式}:
\begin{align}
    \min_{x,y} \quad & c^T x + d^T y \\
    \text{s.t.} \quad & Ax + By \leq b \\
    & x \in \mathbb{R}^n, \quad y \in \{0, 1\}^m
\end{align}

ここで,$x$は連続変数,$y$は二値変数,$c, d$はコスト係数,$A, B$は制約行列,$b$は右辺定数である.

MILPは以下の問題に適用される:
\begin{itemize}
    \item \textbf{排他的選択}:「充電か放電のいずれか一方のみ」のような二者択一の制約
    \item \textbf{固定費用}:ある閾値を超えた場合にのみ発生するコスト
    \item \textbf{論理条件}:if-then形式の条件分岐
\end{itemize}

本研究では,蓄電池の充放電が同時に発生しない(排他制御)という制約を二値変数で表現するためにMILPを採用した.ソルバーには,分枝限定法(Branch and Bound)を基盤とするオープンソースソルバーSCIPを使用した.

\subsection{本研究への適用}

時間軸を$\Delta t = 0.5$時間(30分)間隔で離散化し,各時刻をステップ$k \in \mathbb{Z}_{\geq 0}$で表現する.ローリング計画法では,現在時刻$k$から予測期間$H$ステップ先までの最適化問題を解き,得られた解のうち直近の第1ステップ(時刻$k$の制御入力)のみを実行する.次ステップ$k+1$においては,最新のシステム状態(蓄電池SOC等)および予測されるPV発電量・需要に基づき,再び$H$ステップ先までの計画を更新する.
本手法の採用により,年間を通した大域的な最適化(全17,520ステップを一括で解く場合)に伴う膨大な計算負荷を,各ステップで$H$ステップ分の小規模な問題に分割することで実用的な範囲に低減できる.また,逐次的に計画を更新するため,予測誤差の影響を最小限に抑えたフィードバック制御的な運用が可能となる.本研究では,予測期間$H$および再計画間隔$m$を変化させて比較を行う.

なお,本研究では\textbf{再計画間隔}$m$を導入する.再計画間隔とは,1回の最適化計算で実際に採用・実行する決定ステップ数を指す.基本設定では$m=1$(30分ごとに再計画)であるが,$m$を増加させることで計画頻度を下げ,計算負荷を削減できる.例えば$m=8$とすると,4時間ごとに再計画を行う.予測期間$H$と再計画間隔$m$の関係を以下に示す:

\begin{itemize}
    \item \textbf{予測期間$H$}:最適化問題の時間的視野(何ステップ先まで考慮するか)
    \item \textbf{再計画間隔$m$}:最適化結果のうち実行するステップ数(再計画の頻度に関係する)
\end{itemize}

\noindent
一般に$m \leq H$であり,$m=1$が最も頻繁な再計画(高精度・高計算負荷),$m=H$が最も少ない再計画(低精度・低計算負荷)に対応する.図\ref{fig:rolling_horizon}にローリング計画法の概念図を示す.

\begin{figure}[H]
\centering
\begin{tikzpicture}[scale=0.7]
    % ステップ1
    \node[anchor=east] at (-0.5, 3) {\small ステップ1};
    \draw[->, thick] (0,3) -- (12,3);
    \draw[thick] (0.5,3.2) -- (0.5,2.8) node[below] {\tiny $k$};
    \draw[thick] (2.0,3.2) -- (2.0,2.8) node[below] {\tiny $k\!+\!m$};
    \draw[thick] (8.5,3.2) -- (8.5,2.8) node[below] {\tiny $k\!+\!H$};
    \fill[blue!40] (0.5,2.9) rectangle (2.0,3.1);
    \fill[red!20] (2.0,2.9) rectangle (8.5,3.1);
    \node at (1.25,3) {\tiny 実行};
    \node at (5.25,3) {\tiny 計画(破棄)};

    % ステップ2
    \node[anchor=east] at (-0.5, 1.5) {\small ステップ2};
    \draw[->, thick] (0,1.5) -- (12,1.5);
    \draw[thick] (2.0,1.7) -- (2.0,1.3) node[below] {\tiny $k\!+\!m$};
    \draw[thick] (3.5,1.7) -- (3.5,1.3) node[below] {\tiny $k\!+\!2m$};
    \draw[thick] (10.0,1.7) -- (10.0,1.3) node[below] {\tiny $k\!+\!m\!+\!H$};
    \fill[gray!30] (0.5,1.4) rectangle (2.0,1.6);
    \fill[blue!40] (2.0,1.4) rectangle (3.5,1.6);
    \fill[red!20] (3.5,1.4) rectangle (10.0,1.6);
    \node at (1.25,1.5) {\tiny 完了};
    \node at (2.75,1.5) {\tiny 実行};
    \node at (6.75,1.5) {\tiny 計画(破棄)};

    % ステップ3
    \node[anchor=east] at (-0.5, 0) {\small ステップ3};
    \draw[->, thick] (0,0) -- (12,0);
    \draw[thick] (3.5,0.2) -- (3.5,-0.2) node[below] {\tiny $k\!+\!2m$};
    \draw[thick] (5.0,0.2) -- (5.0,-0.2) node[below] {\tiny $k\!+\!3m$};
    \draw[thick] (11.5,0.2) -- (11.5,-0.2) node[below] {\tiny $k\!+\!2m\!+\!H$};
    \fill[gray!30] (0.5,-0.1) rectangle (3.5,0.1);
    \fill[blue!40] (3.5,-0.1) rectangle (5.0,0.1);
    \fill[red!20] (5.0,-0.1) rectangle (11.5,0.1);
    \node at (2,0) {\tiny 完了};
    \node at (4.25,0) {\tiny 実行};
    \node at (8.25,0) {\tiny 計画(破棄)};

    % 矢印(ステップ間の遷移)
    \draw[->, thick, dashed] (2.0, 2.6) -- (2.0, 1.9);
    \draw[->, thick, dashed] (3.5, 1.1) -- (3.5, 0.4);

    % 凡例
    \fill[blue!40] (0,-1.5) rectangle (0.5,-1.3);
    \node[anchor=west] at (0.6,-1.4) {\small 実行区間($m$ステップ)};
    \fill[red!20] (5,-1.5) rectangle (5.5,-1.3);
    \node[anchor=west] at (5.6,-1.4) {\small 計画区間(次回更新で破棄)};
    \fill[gray!30] (0,-2.0) rectangle (0.5,-1.8);
    \node[anchor=west] at (0.6,-1.9) {\small 実行済み};

    % 本研究での範囲
    \node[anchor=west] at (0,-2.7) {\small \textbf{本研究での変化範囲}:$H$: 24時間〜21日間,$m$: 30分〜12時間};
\end{tikzpicture}
\caption{ローリング計画法の概念図.予測期間$H$の計画を立て,再計画間隔$m$ステップ分のみ実行し,新たな情報に基づき再計画する.本研究では$H$と$m$を変化させて比較を行う.}
\label{fig:rolling_horizon}
\end{figure}

\subsection{料金体系}

本研究で比較検討する2種類の料金体系について説明する.

\subsubsection{北海道電力基本プラン(高圧電力,一般料金)}

北海道電力の料金体系を表\ref{tab:hokkaido_tariff}に示す.

\begin{table}[H]
\centering
\caption{北海道電力の料金体系(2024年4月1日実施)}
\label{tab:hokkaido_tariff}
\begin{tabular}{lc}
\toprule
項目 & 料金単価 \\
\midrule
基本料金 & 2,829.60 円/kW \\
電力量料金 & 21.51 円/kWh \\
再エネ賦課金 & 3.98 円/kWh \\
\bottomrule
\end{tabular}
\end{table}

\textbf{基本料金の計算式:}
\begin{equation}
C_{\mathrm{basic}} = P_{\mathrm{contract}} \times 2829.60 \times 0.85 \times 12 \quad \text{[円/年]}
\end{equation}

ここで,$P_{\mathrm{contract}}$は契約電力であり,過去1年間の各月の最大需要電力のうち,最も大きい値を適用する.2829.60は基本料金単価(円/kW),0.85は力率割引係数,12は年間月数である.本シミュレーションでは,1年間の運用結果から得られた最大買電電力を$P_{\mathrm{contract}}$として事後的に計算している.

\textbf{電力量料金の計算式:}
\begin{equation}
C_{\mathrm{energy}} = E_{\mathrm{month}} \times (21.51 + F_{\mathrm{adj}}(m) + 3.98) \quad \text{[円/月]}
\end{equation}

ここで,$E_{\mathrm{month}}$は月間電力使用量 [kWh],21.51は電力量料金単価(円/kWh),$F_{\mathrm{adj}}(m)$は$m$月($m \in \{1,2,\dots,12\}$)の燃料費調整額(円/kWh),3.98は再生可能エネルギー発電促進賦課金(円/kWh)である.

2024年の月別燃料費調整額を表\ref{tab:fuel_adjustment}に示す.

\begin{table}[H]
\centering
\caption{2024年の月別燃料費調整額(北海道電力・高圧)}
\label{tab:fuel_adjustment}
\begin{tabular}{cc}
\toprule
月 & 燃料費調整額 [円/kWh] \\
\midrule
1月 & $-8.76$ \\
2月 & $-8.59$ \\
3月 & $-8.56$ \\
4月 & $-8.85$ \\
5月 & $-9.02$ \\
6月 & $-7.47$ \\
7月 & $-5.69$ \\
8月 & $-5.69$ \\
9月 & $-9.60$ \\
10月 & $-9.47$ \\
11月 & $-8.06$ \\
12月 & $-5.83$ \\
\bottomrule
\end{tabular}
\end{table}

\subsubsection{市場価格連動プラン}

市場価格連動プランでは,電力量料金がJEPX(日本卸電力取引所)のスポット価格に連動する.

\textbf{電力量料金の計算式:}
\begin{equation}
C_{\mathrm{energy}} = E_{\mathrm{month}} \times (P_{\mathrm{JEPX}}(t) + 3.98) \quad \text{[円/月]}
\end{equation}

ここで,$P_{\mathrm{JEPX}}(t)$は時刻 $t$ のJEPXスポット価格(円/kWh),3.98は再生可能エネルギー発電促進賦課金(円/kWh)である.基本料金は北海道電力と同額とする.

\subsection{数理モデルの定式化}

本節で使用する記号を表\ref{tab:symbols}にまとめる.

\begin{table}[H]
\centering
\caption{記号一覧}
\label{tab:symbols}
\begin{tabular}{clc}
\toprule
記号 & 説明 & 単位 \\
\midrule
\multicolumn{3}{l}{\textbf{決定変数}} \\
$s^{\mathrm{BY}}_{k}$ & 時刻$k$の買電電力 & kW \\
$s^{\mathrm{BY}}_{\mathrm{MAX}}$ & 予測期間内の最大買電電力 & kW \\
$x^{\mathrm{FC1}}_{k}$, $x^{\mathrm{FC2}}_{k}$ & 充電電力(変換前/後) & kW \\
$x^{\mathrm{FD1}}_{k}$, $x^{\mathrm{FD2}}_{k}$ & 放電電力(変換前/後) & kW \\
$b^{\mathrm{F}}_{k}$ & 蓄電池残量(SOC) & kWh \\
$z_{k}$ & 充放電モード(1:充電, 0:放電) & -- \\
$g^{\mathrm{P2}}_{k}$ & 実際に使用するPV発電量 & kW \\
\midrule
\multicolumn{3}{l}{\textbf{既知パラメータ}} \\
$g^{\mathrm{P1}}_{k}$ & PV発電可能量 & kW \\
$d_{k}$ & 需要電力 & kW \\
$p^{\mathrm{BY}}_{k}$ & 電力量単価 & 円/kWh \\
\midrule
\multicolumn{3}{l}{\textbf{定数}} \\
$H$ & 予測期間 & ステップ \\
$\Delta t$ & 1ステップあたりの時間(= 0.5) & h \\
$\eta$ & 充放電効率(= 0.98) & -- \\
$C_{\mathrm{bat}}$ & 蓄電池定格容量 & kWh \\
$P^{\mathrm{FC}}_{\mathrm{max}}$, $P^{\mathrm{FD}}_{\mathrm{max}}$ & 最大充放電出力(= 400) & kW \\
$C_{\mathrm{cap}}$ & 年間基本料金単価 & 円/kW \\
$w_{\mathrm{basic}}$ & 基本料金重み係数 & 円/kW \\
$M$ & Big-M定数(= $10^6$) & -- \\
\bottomrule
\end{tabular}
\end{table}

\subsubsection{決定変数}

最適化問題における決定変数は表\ref{tab:symbols}の通りである.連続変数は非負とし,添字$k$は時刻ステップ($k = 0, 1, \ldots, H-1$)を表す.

既知パラメータとして,PV発電可能量$g^{\mathrm{P1}}_{k}$および需要電力$d_{k}$を与える.出力抑制を許容し,$g^{\mathrm{P2}}_{k} \leq g^{\mathrm{P1}}_{k}$とする.

\subsubsection{目的関数}

日本の高圧電力契約では,電気料金は以下の2つから構成される:
\begin{itemize}
    \item \textbf{基本料金}:契約電力(過去1年間の最大需要電力)に基づく固定費
    \item \textbf{電力量料金}:実際の使用電力量に応じた従量費
\end{itemize}

\noindent
目的関数は,予測期間$H$における電気料金(基本料金相当額および電力量料金)の総和とし,これを最小化する.

\begin{equation}
\text{Minimize} \quad J = \underbrace{w_{\mathrm{basic}} \cdot s^{\mathrm{BY}}_{\mathrm{MAX}}}_{\text{基本料金相当額}} + \underbrace{\sum_{k=0}^{H-1} p^{\mathrm{BY}}_{k} \cdot s^{\mathrm{BY}}_{k} \cdot \Delta t}_{\text{電力量料金}}
\label{eq:objective}
\end{equation}

\noindent
\textbf{目的関数の設計に関する考察:}
本研究の目的関数は各予測期間$H$内のコスト最小化であり,年間電気料金を直接最小化するものではない.契約電力は年間の最大需要電力で決まるため,本来は全17,520ステップを一括最適化すべきであるが,計算負荷を考慮してローリング計画法を採用した.

この制約により,予測期間が短いと基本料金項の重みが相対的に小さくなり,契約電力削減のインセンティブが弱まる.一方,ローリング計画法には,予測誤差の蓄積防止,計算負荷の抑制,異常事態への即応性という利点がある.したがって,本研究の結果は各プランの傾向を示す参考値として解釈すべきである.

ここで,$p^{\mathrm{BY}}_{k}$は時刻$k$における電力量単価 [円/kWh],$w_{\mathrm{basic}}$は基本料金に関する重み係数 [円/kW]である.

基本料金は本来,年間の最大需要電力に基づいて決定されるが,本手法では予測期間が限定的であるため,年間の基本料金単価を時間按分した値をペナルティ項として導入した.重み係数$w_{\mathrm{basic}}$は式(\ref{eq:weight_basic})により算出される.

\begin{equation}
w_{\mathrm{basic}} = C_{\mathrm{cap}} \times \frac{H \cdot \Delta t}{8760}
\label{eq:weight_basic}
\end{equation}

ここで,8760は1年間の時間数($365 \times 24$時間)であり,年間基本料金を予測期間の長さに応じて按分している.

ただし,$C_{\mathrm{cap}}$は年間基本料金単価であり,表\ref{tab:hokkaido_tariff}の料金単価に基づき以下のように算出される:
\begin{equation}
C_{\mathrm{cap}} = 2829.60 \times 0.85 \times 12 \quad \text{[円/kW]}
\end{equation}
ここで,2829.60円/kWは基本料金単価,0.85は力率割引係数(力率85\%以上で適用),12は年間月数である.この項の導入により,各予測期間における買電電力の最大値の抑制を図り,間接的に年間契約電力の低減を指向する.

\noindent
\textbf{注}:本手法は「局所的な最大値の抑制」を通じて「大域的な最大値(年間契約電力)」を間接的に制御する近似であり,その限界については考察(\ref{sec:wbasic_limitation}節)で詳述する.

\subsubsection{制約条件}

システムの物理的特性および運用上の要請に基づき,以下の制約条件を課す.

\noindent
\textbf{(1) 電力需給バランス制約}

各時刻において,供給と需要は一致しなければならない.
\begin{equation}
g^{\mathrm{P2}}_{k} + s^{\mathrm{BY}}_{k} + x^{\mathrm{FD2}}_{k} = d_{k} + x^{\mathrm{FC1}}_{k}, \quad \forall k
\label{eq:balance}
\end{equation}

充放電電力は変換効率$\eta = 0.98$を介して以下の関係にある.
\begin{align}
x^{\mathrm{FC2}}_{k} &= \eta \cdot x^{\mathrm{FC1}}_{k} \label{eq:fc_eff} \\
x^{\mathrm{FD2}}_{k} &= \eta \cdot x^{\mathrm{FD1}}_{k} \label{eq:fd_eff}
\end{align}

ここで,$x^{\mathrm{FC1}}_{k}$は蓄電池への入力電力(変換前),$x^{\mathrm{FC2}}_{k}$は蓄電池に実際に蓄積される電力(変換後)を表す.放電についても同様に,$x^{\mathrm{FD1}}_{k}$は蓄電池から取り出す電力(変換前),$x^{\mathrm{FD2}}_{k}$は負荷で利用可能な電力(変換後)である.なお,本システムでは逆潮流を禁止(売電量ゼロ)とする.

\noindent
\textbf{(2) 蓄電池状態遷移および容量制約}

蓄電池のSOC推移は次式で記述される.
\begin{equation}
b^{\mathrm{F}}_{k+1} = b^{\mathrm{F}}_{k} + (x^{\mathrm{FC2}}_{k} - x^{\mathrm{FD1}}_{k}) \cdot \Delta t
\label{eq:soc_update}
\end{equation}

また,過充電・過放電防止のため,運用範囲を定格容量$C_{\mathrm{bat}}$の5\%〜95\%に制限する.
\begin{equation}
0.05 \cdot C_{\mathrm{bat}} \leq b^{\mathrm{F}}_{k} \leq 0.95 \cdot C_{\mathrm{bat}}, \quad \forall k
\label{eq:soc_limit}
\end{equation}

\noindent
\textbf{(3) 充放電排他および出力制約}

充電と放電の同時実行を物理的に排除するため,二値変数$z_{k}$を用いた以下の制約(Big-M法)を設ける.
\begin{align}
x^{\mathrm{FC1}}_{k} &\leq M \cdot z_{k}, \quad \forall k \label{eq:bigm_fc} \\
x^{\mathrm{FD1}}_{k} &\leq M \cdot (1 - z_{k}), \quad \forall k \label{eq:bigm_fd}
\end{align}

ここで,$M$は十分大きな定数($M = 10^6$)である.また,充放電出力の上限として以下を課す.
\begin{align}
x^{\mathrm{FC2}}_{k} &\leq P^{\mathrm{FC}}_{\mathrm{max}}, \quad \forall k \label{eq:fc_limit} \\
x^{\mathrm{FD1}}_{k} &\leq P^{\mathrm{FD}}_{\mathrm{max}}, \quad \forall k \label{eq:fd_limit}
\end{align}

本研究では$P^{\mathrm{FC}}_{\mathrm{max}} = P^{\mathrm{FD}}_{\mathrm{max}} = 400$\,kWとした.

\noindent
\textbf{(4) PV出力制約}

実際に使用するPV発電量は発電可能量を超えない.
\begin{equation}
g^{\mathrm{P2}}_{k} \leq g^{\mathrm{P1}}_{k}, \quad \forall k
\label{eq:pv_limit}
\end{equation}

\noindent
\textbf{(5) 契約電力制約}

予測期間内のすべての時刻において,買電電力は$s^{\mathrm{BY}}_{\mathrm{MAX}}$以下でなければならない.
\begin{equation}
s^{\mathrm{BY}}_{k} \leq s^{\mathrm{BY}}_{\mathrm{MAX}}, \quad \forall k
\label{eq:contract}
\end{equation}

\section{実験設定}

\subsection{使用データ}

本シミュレーションでは,以下の実測データおよび市場価格データを使用した.

\begin{itemize}
    \item \textbf{対象期間}:2024年1月1日から同年12月31日までの365日間.2024年は閏年であるが,データの整合性を考慮し2月29日を除外した全17,520ステップ(30分時間分解能)を解析対象とした.
    \item \textbf{電力需要およびPV発電量}:北海道十勝地方の対象施設における実測値(30分積算値)を使用した.
    \item \textbf{電力市場価格}:一般社団法人日本卸電力取引所(JEPX)が公開する北海道エリアのスポット市場価格(30分値)を採用した.
\end{itemize}

\subsection{データ前処理}

取得データは30分間の積算電力量$E_{30\mathrm{min}}$\,[kWh]であるため,最適化計算の入力とするにあたり,以下の関係式を用いて平均電力$P_{\mathrm{avg}}$\,[kW]に換算した.
\begin{equation}
P_{\mathrm{avg}} = \frac{E_{30\mathrm{min}}}{0.5}
\label{eq:power_conversion}
\end{equation}

なお,目的関数における電気料金の算出に際しては,決定変数(電力\,[kW])に時間刻み$\Delta t = 0.5$\,[h]を乗じることで,再び電力量\,[kWh]ベースに換算して評価を行っている.

\subsection{計算環境と実装}

アルゴリズムの実装にはPython 3.xを使用し,混合整数計画問題(MILP)のソルバーにはSCIPのPythonインターフェースであるPySCIPOptを採用した.数値計算およびデータ処理にはpandas,numpyライブラリを,結果の可視化にはmatplotlibをそれぞれ用いた.

\subsection{最適化パラメータ}

ローリング計画法における予測期間$H$は,以下のケースについて比較検証を行った:短期予測として$H=48$(24時間),$H=96$(48時間),$H=120$(60時間),$H=144$(72時間)の4ケース,長期予測として$H=192$(4日間),$H=384$(8日間),$H=672$(14日間),$H=1008$(21日間)の4ケースである.再計画間隔$m$は基本的に$m=1$(30分ごとに再計画)とし,長期予測では計算負荷軽減のため$m=8$(4時間)や$m=24$(12時間)も検討した.

\section{結果と考察}

本章では,3つの比較軸(蓄電池容量・予測期間・再計画間隔)に沿ってシミュレーション結果を報告する.いずれの結果も特定の条件下での結論であり,条件によって最適なプランが異なる点に留意されたい.

\subsection{システムの基本的な運用パターン}

需給条件の異なる代表的な日における運用挙動を図\ref{fig:battery_operation}および図\ref{fig:battery_operation_low}に示す.これらの図は蓄電池容量860kWh,北海道電力基本プランの条件下での結果である.

\noindent
\textbf{高需要日}:6月2日(需要2,436kWh,PV発電1,433kWh)と6月24日(需要2,461kWh,PV発電237kWh)を比較する.PV発電が豊富な6月2日は,昼間の余剰電力を蓄電し夜間に放電することで,買電量を1,275kWh(需要の52.4\%)に抑制した.一方,PV発電が少ない6月24日は買電量が2,080kWh(需要の84.5\%)に増加し,買電依存度の差は1.6倍となった.

\noindent
\textbf{低需要日}:2月5日(需要1,284kWh,PV発電1,019kWh)と1月22日(需要1,298kWh,PV発電281kWh)を比較する.PV発電が豊富な2月5日は買電量が270kWh(需要の21.0\%)まで減少し,高い自給率を実現した.一方,PV発電が少ない1月22日は買電量が862kWh(需要の66.4\%)となり,買電依存度の差は3.2倍に達した.この結果から,低需要期においてはPV発電量の多い・少ないが買電量に与える影響が高需要期よりも顕著であることが確認された.

\begin{figure}[H]
\centering
\includegraphics[width=\textwidth]{../png/soc860/daily_battery_pattern.png}
\caption{需要が高い日の運用パターン(需要約2,450 kWh).上:PV発電量が多い日(2024年6月2日,PV発電1,433 kWh),下:PV発電量が少ない日(2024年6月24日,PV発電237 kWh).}
\label{fig:battery_operation}
\end{figure}

\begin{figure}[H]
\centering
\includegraphics[width=\textwidth]{../png/soc860/daily_battery_pattern_low_demand.png}
\caption{需要が低い日の運用パターン(需要約1,290 kWh).上:PV発電量が多い日(2024年2月5日,PV発電1,019 kWh),下:PV発電量が少ない日(2024年1月22日,PV発電281 kWh).}
\label{fig:battery_operation_low}
\end{figure}

また,図\ref{fig:battery_operation}や図\ref{fig:battery_operation_low}において買電電力が一定値を維持する挙動が観察されるが,これは式(\ref{eq:objective})で導入した基本料金項の影響である.契約電力の増大を回避しつつ,蓄電池の充放電で需給調整を行う最適制御が機能していることを示す.

\subsection{蓄電池容量と料金プランの経済性評価}

本節では,予測期間$H=96$(48時間),再計画間隔$m=1$(30分)の条件下において,\textbf{蓄電池容量}を変化させた際の経済性を比較する(図\ref{fig:axis_battery}).なお,この条件は複数の検証ケースのうちの1つであり,予測期間が異なれば結論が変わりうる点に留意されたい(5.4節,5.5節で詳述).

\begin{figure}[H]
\centering
\begin{tikzpicture}[scale=0.6]
    \node[draw, rectangle, minimum width=2.5cm, minimum height=0.8cm] (battery) at (0,0) {蓄電池容量};
    \node[draw, rectangle, minimum width=2.5cm, minimum height=0.8cm, fill=gray!20] (horizon) at (4,0) {予測期間$H$};
    \node[draw, rectangle, minimum width=2.5cm, minimum height=0.8cm, fill=gray!20] (interval) at (8,0) {再計画間隔$m$};
    \draw[->, very thick, red] (0,-0.8) -- (0,-1.5) node[below] {\small \textbf{変化}};
    \node at (4,-1.2) {\small 固定(48h)};
    \node at (8,-1.2) {\small 固定(30分)};
\end{tikzpicture}
\caption{本節の比較軸:蓄電池容量を0〜1720kWhで変化させ,他の条件は固定.}
\label{fig:axis_battery}
\end{figure}

\subsubsection{蓄電池容量による経済性の変化}

蓄電池容量ごとの年間コスト比較を表\ref{tab:capacity_plan_comparison}および表\ref{tab:optimal_comparison}に示す.

\begin{table}[H]
\centering
\caption{蓄電池容量別の年間コスト比較(両プラン)}
\label{tab:capacity_plan_comparison}
\begin{tabular}{rrrrl}
\toprule
容量 [kWh] & 北電基本 [円] & 市場連動 [円] & 差額 [円] & 有利なプラン \\
\midrule
0 & 18,101,486 & 17,422,732 & $-678,754$ & 市場連動 \\
215 & 15,585,445 & 14,787,091 & $-798,354$ & 市場連動 \\
430 & 14,941,626 & 14,613,390 & $-328,236$ & 市場連動 \\
540 & 14,731,394 & 14,859,816 & $-128,422$ & 市場連動 \\
645 & 14,561,085 & 14,858,066 & $-296,981$ & 市場連動 \\
860 & 14,343,827 & 15,023,532 & $-679,705$ & 市場連動 \\
1290 & 14,074,455 & 15,125,066 & $-1,050,611$ & 市場連動 \\
1720 & 14,072,576 & 15,290,495 & $-1,217,919$ & 市場連動 \\
\bottomrule
\end{tabular}
\end{table}

\noindent
差額は「北電基本 $-$ 市場連動」を示す.正の値は北電基本プランが安価,負の値は市場連動プランが安価であることを意味する.

\noindent
\textbf{料金プランの経済性}:全ての蓄電池容量において市場価格連動プランが経済的に有利であった.ただし,蓄電池容量540kWh以上では両プランの差額が縮小し,容量1720kWhでは差額が約122万円まで拡大する.

各料金プランにとって最適な蓄電池容量での比較を表\ref{tab:optimal_comparison}に示す.

\begin{table}[H]
\centering
\caption{各プランで最もコストが低い容量での年間コスト比較}
\label{tab:optimal_comparison}
\begin{tabular}{llrr}
\toprule
料金プラン & コスト最小容量 & 年間コスト [円] & 契約電力 [kW] \\
\midrule
市場価格連動プラン & 430 kWh & 14,613,390 & 201.40 \\
北海道電力基本プラン & 1720 kWh & 14,072,576 & 157.40 \\
\midrule
\multicolumn{2}{l}{\textbf{差額}} & \multicolumn{2}{l}{\textbf{540,814円(北電基本が安価)}} \\
\bottomrule
\end{tabular}
\end{table}

\noindent
\textbf{コスト最小容量での比較}:検討した蓄電池容量(0~1720kWh)の範囲で各プランのコストが最小となる容量(市場連動:430kWh,北電基本:1720kWh)を比較すると,北電基本プランが年間約54万円安価である.なお,本比較は離散的な容量設定に基づく.

\subsubsection{契約電力と運用特性の違い}

両プランの蓄電池容量別詳細比較を表\ref{tab:capacity_comparison_detail}および表\ref{tab:capacity_comparison_market_detail}に示す.表中の「PV利用率」は「実際に使用したPV発電量 $\div$ PV発電可能量 $\times$ 100[\%]」で定義される.蓄電池がない場合,PV余剰電力を消費できず出力抑制が発生するため100\%未満となる.

\begin{table}[H]
\centering
\caption{蓄電池容量別の詳細比較(北海道電力基本プラン)}
\label{tab:capacity_comparison_detail}
\begin{tabular}{rrrrrr}
\toprule
容量 [kWh] & 契約電力 [kW] & 買電量 [kWh] & PV利用率 [\%] & 年間コスト [円] & コスト差 [円] \\
\midrule
0 & 262.00 & 590,587 & 78.01 & 18,101,486 & - \\
215 & 198.01 & 551,647 & 92.43 & 15,585,445 & $-2,516,041$ \\
430 & 185.50 & 535,043 & 98.41 & 14,941,626 & $-3,159,860$ \\
540 & 179.76 & 532,389 & 99.46 & 14,731,394 & $-3,370,092$ \\
645 & 174.36 & 531,536 & 99.86 & 14,561,085 & $-3,540,401$ \\
860 & 166.83 & 531,528 & 100.0 & 14,343,827 & $-3,757,659$ \\
1290 & 157.40 & 531,672 & 100.0 & 14,074,455 & $-4,027,031$ \\
1720 & 157.40 & 531,553 & 100.0 & 14,072,576 & $-4,028,910$ \\
\bottomrule
\end{tabular}
\end{table}

\begin{table}[H]
\centering
\caption{蓄電池容量別の詳細比較(市場価格連動プラン)}
\label{tab:capacity_comparison_market_detail}
\begin{tabular}{rrrrrr}
\toprule
容量 [kWh] & 契約電力 [kW] & 買電量 [kWh] & PV利用率 [\%] & 年間コスト [円] & コスト差 [円] \\
\midrule
0 & 262.00 & 590,587 & 78.01 & 17,422,732 & - \\
215 & 198.01 & 552,550 & 92.43 & 14,787,091 & $-2,635,641$ \\
430 & 201.40 & 535,623 & 98.41 & 14,613,390 & $-2,809,342$ \\
540 & 211.41 & 532,738 & 99.46 & 14,859,816 & $-2,562,916$ \\
645 & 212.00 & 531,684 & 99.86 & 14,858,066 & $-2,564,666$ \\
860 & 218.05 & 531,358 & 100.0 & 15,023,532 & $-2,399,200$ \\
1290 & 221.65 & 531,333 & 100.0 & 15,125,066 & $-2,297,666$ \\
1720 & 227.63 & 531,135 & 100.0 & 15,290,495 & $-2,132,237$ \\
\bottomrule
\end{tabular}
\end{table}

コスト差は蓄電池容量0kWh(蓄電池なし)を基準とする.

図\ref{fig:capacity_contract_power}に蓄電池容量と契約電力・年間コストの関係を示す.

\begin{figure}[H]
\centering
\includegraphics[width=0.9\textwidth]{../png/soc860/capacity_contract_power.png}
\caption{蓄電池容量と契約電力・年間コストの関係}
\label{fig:capacity_contract_power}
\end{figure}

図\ref{fig:pareto_frontier}に,パレートフロンティア曲線として契約電力と電力量料金のトレードオフ関係を示す.各蓄電池容量のプロットがどのような曲線を描くかで,システムの効率性を議論できる.

\begin{figure}[H]
\centering
\includegraphics[width=0.9\textwidth]{../png/thesis_figures/pareto_frontier.png}
\caption{パレートフロンティア:契約電力 vs 電力量料金のトレードオフ.市場価格連動プランは電力量料金で有利だが契約電力が増大する傾向がある.}
\label{fig:pareto_frontier}
\end{figure}

図\ref{fig:peak_distribution}に,月別のピーク発生回数と発生時刻の分布を示す.ピーク(上位5\%の高需要日)がいつ,どの頻度で発生したかを示すことで,契約電力を決定づける要因を分析できる.

\begin{figure}[H]
\centering
\includegraphics[width=\textwidth]{../png/thesis_figures/peak_distribution.png}
\caption{月別ピーク発生分布(上段)と発生時刻の分布(下段).北電基本プランでは7〜9月にピークが集中しているのに対し,市場連動プランでは深夜帯と夕方ピーク帯でピークが発生している.}
\label{fig:peak_distribution}
\end{figure}

図\ref{fig:daily_peak_comparison}に,日最大買電電力の年間推移を両プランで比較して示す.

\begin{figure}[H]
\centering
\includegraphics[width=\textwidth]{../png/thesis_figures/daily_peak_comparison.png}
\caption{日最大買電電力の年間推移(蓄電池860kWh).市場連動プランでは夏季を中心にピークが218kWに達するのに対し,北電基本プランでは166.83kWに抑制されている.}
\label{fig:daily_peak_comparison}
\end{figure}

\noindent
\textbf{北海道電力基本プラン}:蓄電池容量の増加に伴い,契約電力は単調減少した(蓄電池なし時の262.00kWから1720kWh時の157.40kWまで).電力量単価が一定のため,買電タイミングを変えても従量料金は削減されない.したがって,蓄電池の唯一の経済的価値は買電電力の最大値抑制となり,蓄電池がこの目的に最大限活用される.

\noindent
\textbf{市場価格連動プラン}:契約電力は蓄電池容量215kWhのとき198.01kWで最小となり,それ以降は容量増加に伴い逆に増大する傾向を示した(227.63kWまで上昇).これは,価格が安い時間帯に集中的な買電・充電が行われたことに起因する.

\subsubsection{プラン優位性逆転の臨界点:相転移的考察}
\label{sec:phase_transition}

表\ref{tab:capacity_plan_comparison}に示した通り,蓄電池容量430kWhでは市場価格連動プランが約33万円有利であるが,540kWh以上では北海道電力基本プランが有利となり,容量の増加に伴いその差は拡大する.本節では,この「540kWh」という臨界点の物理的意味を,システムパラメータとの関連から考察する.

\paragraph{臨界点とシステムパラメータの比較}

540kWhという値を,対象施設のエネルギー収支の観点から特徴づける.本研究の対象施設は以下のパラメータを持つ:

\begin{itemize}
    \item 年間需要:812,982 kWh(日平均 2,227 kWh)
    \item 年間PV発電量:287,633 kWh(日平均 788 kWh)
    \item PV自給率:35.4\%
    \item 日平均PV余剰電力:約 170 kWh(蓄電池なし時の推定値)
    \item 夜間需要(18:00--6:00):日平均 約 930 kWh
\end{itemize}

これらのパラメータと540kWhを比較すると,興味深い関係が浮かび上がる:

\begin{equation}
\underbrace{540 \text{ kWh}}_{\text{臨界容量}} \approx \underbrace{788 \text{ kWh}}_{\text{日平均PV発電量}} \times 0.69 \approx \underbrace{930 \text{ kWh}}_{\text{日平均夜間需要}} \times 0.58
\end{equation}

\noindent
すなわち,540kWhは「1日のPV発電量の約7割」または「夜間需要の約6割」に相当する.この容量は,\textbf{日常的な日内エネルギーシフト(昼間のPV余剰 → 夜間需要)をほぼ飽和させる閾値}と解釈できる.

\paragraph{PV利用率の飽和挙動}

表\ref{tab:capacity_comparison_detail}および表\ref{tab:capacity_comparison_market_detail}のPV利用率を詳細に検討すると,この解釈を裏付けるデータが得られる:

\begin{table}[H]
\centering
\caption{蓄電池容量とPV利用率の関係}
\label{tab:pv_utilization_saturation}
\begin{tabular}{rrrl}
\toprule
容量 [kWh] & PV利用率 [\%] & 限界改善率 [\%/100kWh] & 飽和状態 \\
\midrule
0 & 78.01 & -- & 未飽和 \\
215 & 92.43 & 6.7 & 未飽和 \\
430 & 98.41 & 2.8 & 飽和接近 \\
540 & 99.46 & 0.95 & \textbf{ほぼ飽和} \\
645 & 99.86 & 0.38 & 飽和 \\
860 & 100.0 & 0.06 & 完全飽和 \\
\bottomrule
\end{tabular}
\end{table}

\noindent
ここで限界改善率は,蓄電池容量100kWhあたりのPV利用率向上を示す.540kWhを境に限界改善率が1\%を下回り,PV余剰電力の吸収機能が実質的に飽和することが分かる.

\paragraph{蓄電池機能のモード遷移(Phase Transition)}

以上の分析から,蓄電池容量540kWhを境界として,蓄電池の主機能が質的に変化する「モード遷移」が生じていると解釈できる:

\begin{description}
    \item[モードI($C < 540$ kWh):アービトラージ優位相] \mbox{}\\
    蓄電池容量が日内エネルギーシフトに必要な量を下回る領域.この領域では:
    \begin{itemize}
        \item PV余剰電力の吸収が不完全(PV利用率 $<$ 99.5\%)
        \item 蓄電池は「価格差益(アービトラージ)」の獲得に積極的に活用される
        \item 市場価格連動プランは,安価な時間帯への買電シフトにより電力量料金を削減
        \item 契約電力の増加(201.4 kW)は,電力量料金削減により相殺される
    \end{itemize}

    \item[モードII($C \geq 540$ kWh):ピークカット優位相] \mbox{}\\
    蓄電池容量が日内エネルギーシフトをほぼ飽和させる領域.この領域では:
    \begin{itemize}
        \item PV余剰電力は完全に吸収(PV利用率 $\approx$ 100\%)
        \item 追加的な容量は「ピークカット」にのみ活用可能
        \item 市場価格連動プランでは,安価な時間帯への買電集中がピーク需要を増大
        \item 契約電力の増加(211〜228 kW)が,限界的な電力量料金削減を上回る
    \end{itemize}
\end{description}

この遷移を数式で表現すると:

\begin{equation}
\frac{\partial \text{Cost}}{\partial C} = \underbrace{\frac{\partial \text{Cost}_{\text{energy}}}{\partial C}}_{\text{電力量料金の限界削減}} + \underbrace{\frac{\partial \text{Cost}_{\text{demand}}}{\partial C}}_{\text{基本料金の限界変化}}
\end{equation}

モードIでは第1項(負)が第2項(正または小)を上回り,総コストは減少する.モードIIでは第1項がゼロに近づき,第2項(正)が支配的となるため,市場価格連動プランでは総コストが増加に転じる.

\paragraph{北海道電力基本プランにおける連続的改善}

対照的に,北海道電力基本プランではモード遷移が発生しない.電力量料金が一定(30.56円/kWh)であるため,蓄電池容量が増加しても「安価な時間帯への買電シフト」のインセンティブが存在しない.結果として:

\begin{itemize}
    \item 蓄電池は常にピークカット機能に活用される
    \item 契約電力は容量に対して単調減少(262.0 kW → 157.4 kW)
    \item 追加的な容量は常に正の限界便益(基本料金削減)をもたらす
\end{itemize}

この「モード遷移の有無」が,両プランの容量依存性の本質的な差異である.

\paragraph{臨界点の一般化}

540kWhという具体的な値は本研究の対象施設に固有であるが,臨界点の存在自体は一般化可能である.臨界容量$C^*$は以下の条件で決定される:

\begin{equation}
C^* \approx \min \left( \bar{E}_{\text{PV-surplus}}, \bar{E}_{\text{night}} \right) \times k
\label{eq:critical_capacity}
\end{equation}

\noindent
ここで$\bar{E}_{\text{PV-surplus}}$は日平均PV余剰電力量,$\bar{E}_{\text{night}}$は日平均夜間需要,$k$は安全係数(本研究では$k \approx 3$)である.式(\ref{eq:critical_capacity})は,「日常的なエネルギーシフトに必要な容量」を近似的に与える.

この臨界容量を超えると,変動価格プランでは蓄電池の限界便益が急速に低下し,固定価格プランの相対的優位性が増大する.したがって,\textbf{料金プランの選択は蓄電池容量の設計と不可分}であり,両者を同時に最適化する必要がある.

\subsubsection{年間運用統計(蓄電池容量860kWh)}

蓄電池容量860kWhにおける年間運用実績を表\ref{tab:annual_cost},表\ref{tab:system_stats}および図\ref{fig:pv_buysell},図\ref{fig:annual_soc}に示す.

\begin{table}[H]
\centering
\caption{年間電気料金の比較(蓄電池860kWh,2024年実績)}
\label{tab:annual_cost}
\begin{tabular}{lrr}
\toprule
項目 & 北海道電力基本プラン & 市場価格連動プラン \\
\midrule
基本料金 [円] & 4,814,924 & 6,293,481 \\
電力量料金 [円] & 11,433,157 & 6,615,247 \\
燃料費調整額 [円] & $-4,019,734$ & - \\
再エネ賦課金 [円] & 2,115,480 & 2,114,804 \\
\midrule
\textbf{年間合計 [円]} & \textbf{14,343,827} & \textbf{15,023,532} \\
\midrule
契約電力 [kW] & 166.83 & 218.05 \\
年間削減額 [円] & \multicolumn{2}{c}{679,705(北海道電力基本プランが安価)} \\
\bottomrule
\end{tabular}
\end{table}

\noindent
\textbf{計算条件:}北海道電力基本プランでは電力量料金21.51円/kWh+月別燃料費調整額($-5.83$〜$-9.60$円/kWh)+再エネ賦課金3.98円/kWh,市場価格連動プランではJEPX価格+再エネ賦課金3.98円/kWhを適用した.基本料金単価は$2{,}829.60 \times 0.85 \times 12$円/kW/年である.

2024年の年間システム運用統計を表\ref{tab:system_stats}に示す.なお,PV自給率は「PV発電による消費量 $\div$ 総需要 $\times$ 100 [\%]」,PV利用率は「実際に使用したPV発電量 $\div$ 発電可能量 $\times$ 100 [\%]」で定義される.

\begin{table}[H]
\centering
\caption{年間システム運用統計(2024年)}
\label{tab:system_stats}
\begin{tabular}{lrr}
\toprule
項目 & 北海道電力基本プラン & 市場価格連動プラン \\
\midrule
総需要電力量 [kWh] & 812,982 & 812,982 \\
総PV発電量 [kWh] & 287,633 & 287,633 \\
\quad PV自家消費量 [kWh] & 287,633 & 287,633 \\
\quad PV余剰量 [kWh] & 0 & 0 \\
総買電量 [kWh] & 531,528 & 531,358 \\
\midrule
PV自給率 [\%] & 35.4 & 35.4 \\
PV利用率 [\%] & 100.00 & 100.00 \\
最大買電電力 [kW] & 166.83 & 218.05 \\
平均買電電力 [kW] & 60.68 & 60.66 \\
平均SOC [kWh] & 362.83 (42.2\%) & 356.69 (41.5\%) \\
満充電回数 & 155 & 125 \\
\bottomrule
\end{tabular}
\end{table}

\noindent
年間総需要812,982kWhに対し,両プランともにPV発電量(287,633kWh)の全量を自家消費し,PV自給率35.4\%を達成した.一方で,運用パターンには差異が見られ,北電基本プランは契約電力の抑制(166.83kW)を優先して平準化された運用を行うのに対し,市場連動プランは価格変動に応じた運用を行うため,最大買電電力が218.05kWまで上昇する結果となった.
% そもそも市場連動プランの価格が、北電基本プランの基本価格とどう違うのかがすごく気になった。そこを調べないと考察が正確にできないと思う

年間のPV発電量・買電量・需要の推移を図\ref{fig:pv_buysell}に,年間の蓄電池SOC推移を図\ref{fig:annual_soc}に示す.

\begin{figure}[H]
\centering
\includegraphics[width=\textwidth]{../png/soc860/annual_pv_buy_demand.png}
\caption{年間のPV発電量・買電量・需要の推移(北海道電力基本プラン,蓄電池860kWh)}
\label{fig:pv_buysell}
\end{figure}

\begin{figure}[H]
\centering
\includegraphics[width=\textwidth]{../png/soc860/annual_soc.png}
\caption{年間の蓄電池SOC推移(北海道電力基本プラン,蓄電池860kWh,2024年)}
\label{fig:annual_soc}
\end{figure}

図\ref{fig:carpet_plot}に,SOCと買電電力の年間パターンをヒートマップ(Carpet Plot)として可視化した.横軸は時刻(0〜24時),縦軸は日付(1月〜12月),色はSOCまたは買電電力の値を示す.この形式により,1年間の運用パターン(季節変化,日長変化,突発的な高需要)を「テクスチャ」として直感的に把握できる.

\begin{figure}[H]
\centering
\includegraphics[width=\textwidth]{../png/thesis_figures/carpet_plot_soc_buy.png}
\caption{SOCと買電電力の年間パターン(ヒートマップ).上段:SOC推移,下段:買電電力.左:北海道電力基本プラン,右:市場価格連動プラン.市場連動プランでは昼間のPV発電時間帯と深夜の低価格時間帯で明確な充電パターンが観察される.}
\label{fig:carpet_plot}
\end{figure}

\subsection{予測期間の影響分析}

本節では,蓄電池860kWh,再計画間隔$m=1$(30分)の条件下において,\textbf{予測期間$H$}を変化させた際の影響を分析する(図\ref{fig:axis_horizon}).

\begin{figure}[H]
\centering
\begin{tikzpicture}[scale=0.6]
    \node[draw, rectangle, minimum width=2.5cm, minimum height=0.8cm, fill=gray!20] (battery) at (0,0) {蓄電池容量};
    \node[draw, rectangle, minimum width=2.5cm, minimum height=0.8cm] (horizon) at (4,0) {予測期間$H$};
    \node[draw, rectangle, minimum width=2.5cm, minimum height=0.8cm, fill=gray!20] (interval) at (8,0) {再計画間隔$m$};
    \node at (0,-1.2) {\small 固定(860kWh)};
    \draw[->, very thick, red] (4,-0.8) -- (4,-1.5) node[below] {\small \textbf{変化}};
    \node at (8,-1.2) {\small 固定(30分)};
\end{tikzpicture}
\caption{本節の比較軸:予測期間$H$を24〜72時間で変化させ,他の条件は固定.}
\label{fig:axis_horizon}
\end{figure}

\subsubsection{予測期間と経済効果}

各予測期間における年間統計を表\ref{tab:horizon_comparison}(市場価格連動プラン)および表\ref{tab:horizon_comparison_hokkaido}(北海道電力基本プラン)に示す.

\begin{table}[H]
\centering
\caption{予測期間による年間統計の比較(市場価格連動プラン,蓄電池860kWh)}
\label{tab:horizon_comparison}
\begin{tabular}{lrrrr}
\toprule
項目 & 24時間予測 & 48時間予測 & 60時間予測 & 72時間予測 \\
\midrule
契約電力 [kW] & 230.49 & 218.05 & 218.05 & 213.19 \\
年間買電量 [kWh] & 530,742 & 531,358 & 531,418 & 531,373 \\
年間コスト [円] & 15,404,698 & 15,023,532 & 15,003,341 & 14,847,956 \\
24時間比コスト差 [円] & - & $-$381,166 & $-$401,357 & $-$556,742 \\
\bottomrule
\end{tabular}
\end{table}

\begin{table}[H]
\centering
\caption{予測期間による年間統計の比較(北海道電力基本プラン,蓄電池860kWh)}
\label{tab:horizon_comparison_hokkaido}
\begin{tabular}{lrrrr}
\toprule
項目 & 24時間予測 & 48時間予測 & 60時間予測 & 72時間予測 \\
\midrule
契約電力 [kW] & 169.68 & 166.83 & 166.83 & 166.83 \\
年間買電量 [kWh] & 531,707 & 531,528 & 531,291 & 531,145 \\
年間コスト [円] & 14,429,874 & 14,343,827 & 14,339,597 & 14,337,000 \\
24時間比コスト差 [円] & - & $-$86,047 & $-$90,277 & $-$92,874 \\
\bottomrule
\end{tabular}
\end{table}
\noindent
市場価格連動プランでは,予測期間を24時間から48時間に延長することで年間コストが約38.1万円(2.5\%)削減された.60時間予測では48時間予測からわずか約2.0万円の追加削減,72時間予測では48時間予測から約17.6万円の追加削減となり,\textbf{改善効果の逓減}が確認された.

北海道電力基本プランでは,予測期間延長による効果がより限定的であり,24時間から72時間への延長で約9.3万円(0.6\%)の削減に留まった.これは,電力量単価が一定のため価格変動への追従が不要であり,主に買電電力の最大値抑制の精度向上のみが効果として現れるためである.

\subsubsection{計算コストとのトレードオフ}

予測期間ごとの計算時間を表\ref{tab:horizon_timing}に示す.計算環境はMacBook Pro(Apple M1 Pro, 16GB RAM)である.

\begin{table}[H]
\centering
\caption{予測期間ごとの計算時間(蓄電池860kWh)}
\label{tab:horizon_timing}
\begin{tabular}{lrrr}
\toprule
予測期間 & 北海道電力プラン [分] & 市場連動プラン [分] & 合計 [分] \\
\midrule
$H=48$(24時間) & 7.9 & 31.0 & 39.3 \\
$H=96$(48時間) & 16.1 & 16.4 & 33.2 \\
$H=120$(60時間) & 22.7 & 23.2 & 46.3 \\
$H=144$(72時間) & 32.3 & 27.9 & 60.6 \\
\bottomrule
\end{tabular}
\end{table}

\noindent
計算時間は予測期間の延長に伴い増加した.市場価格連動プランでは,48時間予測は24時間予測と比較して計算時間が短縮される一方,コストは約38.1万円削減され,費用対効果が高い.60時間予測は48時間予測から計算時間が約13.1分増加するが,追加のコスト削減はわずか約2.0万円に留まった.本システム構成と検証した4つの予測期間においては,計算資源と削減効果のバランスから\textbf{48時間(96ステップ)の予測期間が実用的な選択肢の一つ}であると考えられる.

\subsection{年間一括最適化との比較}

本節では,ローリング計画法の理論的限界を評価するため,年間全データ(17,520ステップ)を一括で最適化するシミュレーションを実施し,その結果をローリング計画法と比較する.年間一括最適化は,将来の需要・PV発電・市場価格を完全に予見した上での理論的最適解であり,ローリング計画法による解がどの程度最適に近いかを評価するベンチマークとなる.

\subsubsection{年間一括最適化の実装}

年間一括最適化では,ローリング計画法と同一の目的関数・制約条件を用いつつ,予測期間$H$を年間全ステップ($H=17{,}520$)に設定した.ただし,目的関数における基本料金項の重み係数$w_{\mathrm{basic}}$は年間按分ではなく,年間の最大買電電力が直接契約電力となるよう式(\ref{eq:weight_basic})において$H = 17{,}520$として計算した.計算時間を1時間に制限し,ソルバー(SCIP)が最適解またはそれに準ずる解を発見した時点で終了するよう設定した.

\subsubsection{蓄電池容量別の比較結果}

蓄電池容量430kWh,860kWh,1290kWhの3ケースについて,ローリング計画法(48時間予測)と年間一括最適化の結果を比較した.表\ref{tab:annual_vs_rolling_hokkaido}に北海道電力基本プラン,表\ref{tab:annual_vs_rolling_market}に市場価格連動プランの結果を示す.

\begin{table}[H]
\centering
\caption{ローリング計画法と年間一括最適化の比較(北海道電力基本プラン)}
\label{tab:annual_vs_rolling_hokkaido}
\begin{tabular}{lrrrrr}
\toprule
容量 [kWh] & \multicolumn{2}{c}{年間コスト [万円]} & コスト差 & \multicolumn{2}{c}{契約電力 [kW]} \\
\cmidrule(lr){2-3} \cmidrule(lr){5-6}
 & ローリング & 年間一括 & [万円] (\%) & ローリング & 年間一括 \\
\midrule
430 & 1,494.2 & 1,490.0 & $-$4.2 (0.28\%) & 185.5 & 185.5 \\
860 & 1,434.4 & 1,427.8 & $-$6.6 (0.46\%) & 166.8 & 166.8 \\
1290 & 1,407.4 & 1,397.8 & $-$9.6 (0.69\%) & 157.4 & 156.4 \\
\bottomrule
\end{tabular}
\end{table}

\begin{table}[H]
\centering
\caption{ローリング計画法と年間一括最適化の比較(市場価格連動プラン)}
\label{tab:annual_vs_rolling_market}
\begin{tabular}{lrrrrr}
\toprule
容量 [kWh] & \multicolumn{2}{c}{年間コスト [万円]} & コスト差 & \multicolumn{2}{c}{契約電力 [kW]} \\
\cmidrule(lr){2-3} \cmidrule(lr){5-6}
 & ローリング & 年間一括 & [万円] (\%) & ローリング & 年間一括 \\
\midrule
430 & 1,461.3 & 1,338.9 & $-$122.4 (8.4\%) & 201.4 & 185.5 \\
860 & 1,502.4 & 1,262.9 & $-$239.5 (15.9\%) & 218.1 & 166.8 \\
1290 & 1,512.5 & 1,233.4 & $-$279.1 (18.5\%) & 221.6 & 156.4 \\
\bottomrule
\end{tabular}
\end{table}

\noindent
北海道電力基本プランでは,ローリング計画法と年間一括最適化のコスト差は0.28\%〜0.69\%と小さく,ローリング計画法でもほぼ最適解に近い結果が得られていることが確認された.一方,市場価格連動プランでは8.4\%〜18.5\%と大きな差が生じた.この差異の原因について次節で考察する.

\subsubsection{計算時間の比較}

年間一括最適化の計算時間を表\ref{tab:annual_timing}に示す.

\begin{table}[H]
\centering
\caption{年間一括最適化の計算時間}
\label{tab:annual_timing}
\begin{tabular}{lrr}
\toprule
容量 [kWh] & 北海道電力プラン [分] & 市場連動プラン [分] \\
\midrule
430 & 2.2 & 1.9 \\
860 & 2.0 & 2.5 \\
1290 & 2.6 & 2.2 \\
\bottomrule
\end{tabular}
\end{table}

\noindent
年間一括最適化は約2分で完了し,ローリング計画法(約35〜40分)の約\textbf{20分の1}の計算時間であった.これは,ローリング計画法が17,520回の小規模最適化問題を逐次解くのに対し,年間一括最適化は1回の大規模最適化問題を解くためである.問題規模は大きいが(約87,000変数,87,000制約),現代のMIPソルバー(SCIP)の効率的な前処理とカット生成により,短時間で最適解が発見された.

\subsubsection{料金プラン優位性の再評価}

年間一括最適化の結果に基づき,料金プランの優位性を再評価した(表\ref{tab:plan_comparison_annual}).

\begin{table}[H]
\centering
\caption{年間一括最適化による料金プラン比較}
\label{tab:plan_comparison_annual}
\begin{tabular}{lrrrl}
\toprule
容量 [kWh] & 北電基本 [万円] & 市場連動 [万円] & 差額 [万円] & 有利なプラン \\
\midrule
430 & 1,490.0 & 1,338.9 & $-$151.1 & \textbf{市場連動} \\
860 & 1,427.8 & 1,262.9 & $-$164.9 & \textbf{市場連動} \\
1290 & 1,397.8 & 1,233.4 & $-$164.4 & \textbf{市場連動} \\
\bottomrule
\end{tabular}
\end{table}

\noindent
ローリング計画法では蓄電池容量540kWh以上で北海道電力基本プランが有利となる「優位性逆転」が観察された(表\ref{tab:capacity_plan_comparison})が,年間一括最適化では\textbf{全ての容量において市場価格連動プランが有利}という結果となった.この差異は,ローリング計画法における市場価格連動プランの運用が局所最適に陥っていることを示唆する.

\subsubsection{近視眼的最適化による契約電力増大}
\label{sec:bounded_rationality}

前節で明らかになった市場価格連動プランにおけるローリング計画法の性能劣化(年間一括最適化との乖離8~18\%)のメカニズムを分析する.
本研究のローリング計画法におけるコントローラは,48時間という限られた予測期間内でのみ最適化を行う.北海道電力基本プランでは電力量料金が一定(30.56円/kWh)のため,コントローラの局所的最適行動(買電電力の平準化)が自然に大域的最適(契約電力抑制)と一致する.

一方,市場価格連動プランではJEPX価格の変動(3.80~31.00円/kWh)により,コントローラは「安価な時間帯に買電を集中」という局所的に最適な行動をとる.この行動は48時間内では確かに最適だが,年間を通じて繰り返されると,特定の時間帯(深夜帯など)に買電が集中し,契約電力が増大する.

\begin{equation}
\underbrace{\sum_{t \in \mathcal{H}} \pi_t \cdot p_t^{\mathrm{BY}}}_{\text{電力量料金の最小化(局所目標)}} \quad \text{vs.} \quad \underbrace{\max_{t \in \mathcal{Y}} p_t^{\mathrm{BY}}}_{\text{契約電力の抑制(大域目標)}}
\end{equation}

\noindent
ここで$\mathcal{H}$は予測期間,$\mathcal{Y}$は年間全期間である.エージェントは$\mathcal{H}$内での電力量料金最小化を優先するが,この行動が$\mathcal{Y}$における契約電力増大を招く.

特に問題となるのは,蓄電池容量が大きいほど各時点での「安価電力の買いだめ」能力が高まり,契約電力の増大を招くことである.表\ref{tab:annual_vs_rolling_market}において,蓄電池容量の増加に伴い契約電力が185.5kW(430kWh)から221.6kW(1290kWh)へと増加した現象は,このメカニズムの発現である.

北海道電力基本プランではこの問題が自然に回避される.電力量料金が時間によらず一定のため,「安価な時間帯への買電集中」というインセンティブが存在しないからである.

\subsection{再計画間隔と長期予測期間の影響}

本節では,蓄電池860kWhの条件下において,\textbf{再計画間隔$m$}および\textbf{予測期間$H$}を変化させた際の影響を検証する(図\ref{fig:axis_interval}).

\begin{figure}[H]
\centering
\begin{tikzpicture}[scale=0.6]
    \node[draw, rectangle, minimum width=2.5cm, minimum height=0.8cm, fill=gray!20] (battery) at (0,0) {蓄電池容量};
    \node[draw, rectangle, minimum width=2.5cm, minimum height=0.8cm] (horizon) at (4,0) {予測期間$H$};
    \node[draw, rectangle, minimum width=2.5cm, minimum height=0.8cm] (interval) at (8,0) {再計画間隔$m$};
    \node at (0,-1.2) {\small 固定(860kWh)};
    \draw[->, very thick, red] (4,-0.8) -- (4,-1.5) node[below] {\small \textbf{変化}};
    \draw[->, very thick, red] (8,-0.8) -- (8,-1.5) node[below] {\small \textbf{変化}};
\end{tikzpicture}
\caption{本節の比較軸:予測期間$H$(4〜21日間)と再計画間隔$m$(30分〜12時間)を変化させる.}
\label{fig:axis_interval}
\end{figure}

\subsubsection{再計画間隔の影響}

予測期間96ステップ(48時間)において,再計画間隔を変化させた結果を表\ref{tab:control_horizon}に示す.

\begin{table}[H]
\centering
\caption{再計画間隔の影響(予測期間48時間,蓄電池860kWh)}
\label{tab:control_horizon}
\begin{tabular}{lrrrr}
\toprule
再計画間隔 & 北電基本 [万円] & 市場連動 [万円] & 計算時間 [分] & 時間削減率 \\
\midrule
1ステップ(30分)& 1,434.4 & 1,502.4 & 33.2 & - \\
2ステップ(1時間)& 1,434.4 & 1,502.2 & 15.9 & 52\% \\
4ステップ(2時間)& 1,434.4 & 1,501.6 & 7.9 & 76\% \\
8ステップ(4時間)& 1,434.4 & 1,500.7 & 4.1 & 88\% \\
\bottomrule
\end{tabular}
\end{table}

\noindent
北海道電力基本プランではコストへの影響がなく,市場価格連動プランでも最大0.1\%程度の差に留まった.一方,計算時間は再計画間隔8で基準の約\textbf{12\%}まで短縮された.この結果は,再計画間隔の延長が計算効率の大幅な改善をもたらす一方,解の品質への影響は限定的であることを示す.

\subsubsection{長期予測期間と再計画間隔の組み合わせ}

再計画間隔の延長により計算時間を抑制しつつ,予測期間を192ステップ(4日間),384ステップ(8日間),672ステップ(14日間)まで延長した結果を表\ref{tab:long_horizon}に示す.

\begin{table}[H]
\centering
\caption{長期予測期間と再計画間隔の影響(蓄電池860kWh)}
\label{tab:long_horizon}
\begin{tabular}{llrrrr}
\toprule
予測期間 & 制御H & 北電基本 [万円] & 市場連動 [万円] & 計算時間 [分] & 年間一括比 \\
\midrule
24時間(48) & 1 & 1,443.0 & 1,540.5 & 39.3 & 22.0\% \\
48時間(96) & 1 & 1,434.4 & 1,502.4 & 33.2 & 19.0\% \\
\midrule
4日間(192)& 2 & 1,433.1 & 1,452.7 & 46.4 & 15.0\% \\
4日間(192)& 4 & 1,433.1 & 1,452.5 & 23.1 & 15.0\% \\
4日間(192)& 8 & 1,433.1 & 1,448.3 & 11.1 & 14.7\% \\
\midrule
8日間(384)& 4 & 1,431.5 & 1,420.9 & 53.1 & 12.5\% \\
8日間(384)& 8 & 1,431.5 & 1,420.6 & 25.5 & 12.5\% \\
\midrule
14日間(672)& 8 & 1,430.4 & 1,398.9 & 48.5 & 10.8\% \\
14日間(672)& 16 & 1,430.4 & 1,398.8 & 25.1 & 10.8\% \\
\midrule
21日間(1008)& 24 & 1,429.9 & \textbf{1,380.2} & 27.6 & \textbf{9.3\%} \\
\midrule
\multicolumn{2}{l}{年間一括最適化} & 1,427.8 & 1,262.9 & 4.5 & 0\% \\
\bottomrule
\end{tabular}
\end{table}

\noindent
「年間一括比」は市場価格連動プランにおいて,年間一括最適化との差を示す($(コスト - 1262.9) / 1262.9 \times 100$).

主な知見は以下の通りである:

\begin{enumerate}
    \item \textbf{予測期間延長の効果}:予測期間を14日間(672ステップ)まで延長することで,市場価格連動プランのコストは1,398.9万円まで改善し,年間一括最適化との差は22.0\%から10.8\%まで縮小した.

    \item \textbf{料金プラン優位性の回復}:8日間(384ステップ)以上の予測期間では,市場価格連動プラン(1,420.9万円)が北海道電力基本プラン(1,431.5万円)より\textbf{10.6万円安価}となり,年間一括最適化と同様に市場連動プランの優位性が回復した.

    \item \textbf{計算時間の抑制}:再計画間隔8を用いることで,14日間予測でも計算時間を48.5分に抑制できた.これは48時間予測・再計画間隔1(33.2分)と同程度である.

    \item \textbf{再計画間隔の影響}:同一予測期間では,再計画間隔によるコスト差は0.5万円未満であり,計算効率の改善効果が支配的であった.
\end{enumerate}

\section{考察}

\subsection{料金プランと蓄電池容量の経済性評価}

本節では,シナリオAの結果に基づき,蓄電池容量がある程度大きい場合に北海道電力基本プランが市場価格連動プランより有利となる理由を分析する.

\subsubsection{料金プランの優位性が逆転するメカニズム}

蓄電池容量430kWh以下では市場価格連動プランが有利,540kWh以上では北海道電力基本プランが有利となる逆転現象が確認された(表\ref{tab:capacity_plan_comparison}).この逆転は,両プランにおける蓄電池の最適運用戦略の違いに起因すると考えられる.

\

\textbf{北海道電力基本プラン}:電力量単価が一定(21.51円/kWh)であるため,買電のタイミングによる電力量料金の差は生じない.このため,最適化において電力量料金を削減する動機がなく,基本料金(契約電力)の削減が唯一の最適化目標となる.結果として,蓄電池は買電電力の最大値抑制に最大限活用され,蓄電池容量の増加に伴い契約電力は単調減少し(262.00kW→157.40kW),コスト削減効果が継続する.

\

\textbf{市場価格連動プラン}:JEPX価格が時間帯により変動する(3.80〜31.00円/kWh)ため,安価な時間帯に買電を集中させる時間的裁定が有効となる.しかし,この戦略は買電の最大電力を増大させ,契約電力の上昇を招く.蓄電池容量430kWh以降では,契約電力増加による基本料金の増加が時間的裁定による電力量料金の削減を上回り,コストが増加に転じる(契約電力:201.40kW→227.63kW).

\subsubsection{蓄電池導入による経済効果}

蓄電池導入効果を表\ref{tab:battery_effect}に示す.

\begin{table}[H]
\centering
\caption{蓄電池導入効果の比較(蓄電池0kWh vs 860kWh)}
\label{tab:battery_effect}
\begin{tabular}{llrrr}
\toprule
項目 & 料金プラン & 蓄電池なし & 蓄電池860kWh & 削減額/削減率 \\
\midrule
\multirow{2}{*}{年間コスト [万円]} & 北海道電力 & 1,810.1 & 1,434.4 & 375.7 (20.8\%) \\
 & 市場連動 & 1,742.3 & 1,502.4 & 239.9 (13.8\%) \\
\midrule
\multirow{2}{*}{契約電力 [kW]} & 北海道電力 & 262.0 & 166.8 & 95.2 (36.3\%) \\
 & 市場連動 & 262.0 & 218.1 & 43.9 (16.8\%) \\
\midrule
\multirow{2}{*}{年間買電量 [MWh]} & 北海道電力 & 590.6 & 531.5 & 59.1 (10.0\%) \\
 & 市場連動 & 590.6 & 531.4 & 59.2 (10.0\%) \\
\midrule
PV利用率 [\%] & 両プラン共通 & 78.0 & 100.0 & \\
\bottomrule
\end{tabular}
\end{table}


\noindent
北海道電力基本プランでは年間\textbf{375.7万円(20.8\%)},市場価格連動プランでは年間\textbf{239.9万円(13.8\%)}のコスト削減を達成した.北海道電力基本プランの削減効果がより大きい理由は,契約電力の削減率(36.3\% vs 16.8\%)の差に起因する.

蓄電池なしでは市場価格連動プランが有利(年間67.9万円の差)であったが,蓄電池860kWhでは北海道電力基本プランが有利(年間68.0万円/年)となり,\textbf{蓄電池導入により最適な料金プランが逆転した}.

\subsubsection{最適プランの結論と留意事項}

各プランの最適容量(市場連動:430kWh,北電基本:1720kWh)で比較した場合でも,北海道電力基本プランの方が年間約54万円安価である(表\ref{tab:optimal_comparison}).蓄電池860kWhと北海道電力基本プランの組み合わせは,蓄電池なし・市場価格連動プランと比較して年間\textbf{307.9万円}のコスト削減が可能である.

\noindent
\textbf{留意事項}:上記の比較は年間運用コスト(OPEX)のみに基づいており,蓄電池の初期投資コスト(CAPEX)は考慮していない.最適蓄電池容量が430kWhと1720kWhで約4倍異なるため,実際の投資判断においては蓄電池単価,耐用年数,割引率等を考慮したライフサイクルコスト分析が必要である.

\subsection{季節変動と市場価格特性による要因分析}

本節では,北海道電力基本プランが有利となる背景を,季節別の蓄電池効果と市場価格特性の両面から分析する.

\subsubsection{季節別の蓄電池効果}

蓄電池860kWhの運用データを月別に分析した結果を表\ref{tab:monthly_analysis}に示す.ここで,買電抑制率は$(\text{需要の最大電力} - \text{買電の最大電力}) / \text{需要の最大電力} \times 100$で算出した値であり,蓄電池によって買電の最大電力がどの程度抑制されたかを示す.

\begin{table}[H]
\centering
\caption{月別エネルギー統計と蓄電池効果}
\label{tab:monthly_analysis}
\begin{tabular}{lrrrrrr}
\toprule
月 & 需要 [MWh] & PV [MWh] & 買電 [MWh] & 需要最大 [kW] & 買電最大 [kW] & 買電抑制率 [\%] \\
\midrule
1月 & 40.2 & 20.9 & 19.7 & 130.0 & 41.0 & 68.5 \\
2月 & 36.7 & 25.3 & 12.3 & 150.0 & 29.0 & 80.7 \\
3月 & 49.0 & 32.4 & 16.8 & 154.0 & 36.2 & 76.5 \\
4月 & 54.9 & 26.6 & 29.0 & 220.0 & 64.0 & 70.9 \\
5月 & 66.6 & 27.1 & 39.9 & 204.0 & 80.4 & 60.6 \\
6月 & 77.6 & 24.8 & 53.3 & 246.0 & 121.8 & 50.5 \\
7月 & 111.2 & 25.6 & 86.2 & 260.0 & 154.9 & 40.4 \\
8月 & 118.2 & 18.8 & 99.8 & 264.0 & 166.8 & 36.8 \\
9月 & 90.9 & 24.9 & 66.7 & 242.0 & 139.0 & 42.6 \\
10月 & 72.5 & 22.0 & 50.9 & 218.0 & 115.5 & 47.0 \\
11月 & 49.9 & 19.5 & 30.8 & 144.0 & 61.0 & 57.6 \\
12月 & 45.3 & 19.8 & 25.8 & 150.0 & 43.5 & 71.0 \\
\midrule
年間 & 813.0 & 287.6 & 531.4 & 264.0 & 166.8 & -- \\
\bottomrule
\end{tabular}
\end{table}

図\ref{fig:price_charge_scatter}に,JEPX価格と充放電量の関係を散布図として示す.この図は市場価格連動プランにおいて,最適化アルゴリズムがどのような価格帯で充放電を行っているか,その「ポリシー関数」を可視化したものである.

\begin{figure}[H]
\centering
\includegraphics[width=\textwidth]{../png/thesis_figures/price_charge_scatter.png}
\caption{JEPX価格と充放電電力の関係(市場価格連動プラン,蓄電池860kWh).左:充電電力,右:放電電力.色は月を表す.低価格時に充電,高価格時に放電する明確なポリシーが観察される.}
\label{fig:price_charge_scatter}
\end{figure}

図\ref{fig:price_charge_bar}に,価格帯別の平均充放電電力を示す.これにより,最適化アルゴリズムの価格応答特性が定量的に把握できる.

\begin{figure}[H]
\centering
\includegraphics[width=0.9\textwidth]{../png/thesis_figures/price_charge_bar.png}
\caption{価格帯別の平均充放電電力(市場価格連動プラン).低価格帯(0〜10円/kWh)では充電が活発であり,高価格帯(25円/kWh以上)では放電が増加する傾向が確認される.}
\label{fig:price_charge_bar}
\end{figure}

\noindent
月別分析から,以下の知見が得られた:

\begin{enumerate}
    \item \textbf{冬季(12--2月)の蓄電池効果が最大}:買電抑制率68.5〜80.7\%と高い値を達成した.これは,冬季の需要122.2MWhに対しPV発電65.9MWh(需要の54\%)が確保されており,蓄電池容量860kWhが需要に対して相対的に大きいためである.特に2月は買電抑制率80.7\%と最高値を記録した.

    \item \textbf{夏季(6--8月)の蓄電池効果が最小}:買電抑制率36.8〜50.5\%に留まった.夏季は需要307.0MWhと年間最大である一方,PV発電は69.3MWh(需要の23\%)に過ぎない.8月は需要118.2MWhに対しPV発電18.8MWhと最も需給バランスが悪く,蓄電池容量860kWhでは買電の最大電力を十分に抑制できない.

    \item \textbf{PV発電量の季節パターン}:3月が32.4MWhで最大,8月が18.8MWhで最小となった.

    \item \textbf{契約電力への示唆}:年間の買電最大電力は8月に発生(166.8kW)した.夏季は買電抑制率が低く(項目2参照),かつ需要の最大電力が大きい(8月:264.0kW)ため,抑制後の買電最大電力も高くなる.

    \item \textbf{蓄電池サイクル数}:年間推定218サイクル(月平均18サイクル)であり,一般的なリチウムイオン電池の寿命(3,000〜6,000サイクル)に対して十分な余裕がある.なお,1サイクルとは蓄電池容量相当の電力量を充放電する単位を指し,部分的な充放電は累積して換算する.
\end{enumerate}

図\ref{fig:monthly_analysis}に月別エネルギー量と買電抑制率のグラフを示す.

\begin{figure}[H]
    \centering
    \includegraphics[width=0.95\textwidth]{../png/soc860/monthly_analysis.png}
    \caption{月別エネルギー量と蓄電池効果の分析.左下グラフの「PV自家消費率」は需要に対するPV発電の直接消費割合,「蓄電池放電貢献率」は需要に対する蓄電池放電量の割合を示す.右下グラフの「買電抑制率」は $(\text{需要の最大電力} - \text{買電の最大電力}) / \text{需要の最大電力} \times 100$ で算出.}
    \label{fig:monthly_analysis}
\end{figure}

\subsubsection{市場価格(JEPX)の季節変動特性}

JEPXスポット価格の月別統計を表\ref{tab:price_monthly}に示す.ここで,「$>$北電率」は市場価格が北海道電力基本プランの電力量料金(21.51円/kWh)を上回る時間帯の割合を示す.

\begin{table}[H]
\centering
\caption{月別市場価格統計}
\label{tab:price_monthly}
\begin{tabular}{lrrrrrr}
\toprule
月 & 平均 [円/kWh] & 中央値 [円/kWh] & 最小 [円/kWh] & 最大 [円/kWh] & $>$北電率 [\%] & 高価格回数 \\
\midrule
1月 & 14.07 & 13.79 & 3.99 & 23.10 & 0.4 & 0 \\
2月 & 13.42 & 13.00 & 3.99 & 25.22 & 1.9 & 2 \\
3月 & 15.94 & 15.79 & 3.99 & 40.57 & 11.4 & 16 \\
4月 & 14.30 & 15.52 & 3.99 & 23.98 & 4.4 & 0 \\
5月 & 15.67 & 16.03 & 3.99 & 25.88 & 14.0 & 4 \\
6月 & 16.72 & 16.52 & 3.99 & 25.75 & 15.4 & 12 \\
7月 & 18.39 & 17.48 & 3.99 & 33.98 & 26.1 & 124 \\
8月 & 19.15 & 19.62 & 8.99 & 25.31 & 32.8 & 60 \\
9月 & 19.27 & 17.98 & 3.99 & 53.63 & 30.3 & 104 \\
10月 & 18.75 & 18.05 & 3.99 & 42.98 & 37.6 & 30 \\
11月 & 18.78 & 18.45 & 6.98 & 25.30 & 29.4 & 30 \\
12月 & 17.70 & 18.09 & 4.00 & 26.98 & 16.4 & 18 \\
\bottomrule
\end{tabular}
\end{table}

\noindent
「高価格回数」は25円/kWh以上の価格が発生した30分コマ数を示す.年間最高価格53.63円/kWhは2024年9月20日9:00に発生した.

市場価格分析から,以下の知見が得られた:

\begin{enumerate}
    \item \textbf{市場価格の年間平均は北海道電力より低い}:年間平均16.87円/kWhは北海道電力基本プラン(21.51円/kWh)より\textbf{21.6\%安い}.しかし,この単純比較は買電タイミングを考慮していない.

    \item \textbf{市場価格が北海道電力を上回る時間帯は18.5\%}:年間17,520コマのうち3,236コマで市場価格が21.51円/kWhを超過する.特に秋季(9--11月)は32.5\%と最も高く,冬季(12--2月)は6.4\%と最も低い.

    \item \textbf{高価格の季節集中}:25円/kWh以上の高価格は年間400回(2.3\%)発生し,7月(124回),9月(104回),8月(60回)に集中している.これは冷房需要増加による電力逼迫を反映している.

    \item \textbf{高価格の時間帯集中}:25円/kWh以上の高価格は9:00--14:00に集中しており(上位5時間帯で全体の57.5\%),昼間の需要が高い時間帯と一致する.この時間帯は需要も高いため,市場価格連動プランでは高価格での買電が避けられない.

    \item \textbf{時間帯別価格パターン}:深夜(0:00--2:00)は平均11--12円/kWh,昼間(11:00--14:00)は平均20--21円/kWhと,約2倍の価格差がある.市場価格連動プランはこの価格差を利用した時間シフトが可能だが,需要パターンとの制約により完全な活用は困難である.
\end{enumerate}

\subsubsection{北海道電力基本プランが有利となる理由}

以上の分析から,北海道電力基本プランが市場価格連動プランより年間約64万円安価となる理由は以下のように説明できる:

\begin{enumerate}
    \item \textbf{高価格帯の回避の困難さ}:市場価格連動プランでは,需要が高い時間帯(昼間)に高価格が発生するため,高価格での買電が避けられない.蓄電池による時間シフトを行っても,夏季・秋季の需要が蓄電池容量(860kWh)を大きく上回るため,高価格帯の回避には限界がある.

    \item \textbf{契約電力の増加}:市場価格連動プランでは安価な時間帯に集中的に買電・充電するため,買電の最大電力が上昇し契約電力が増加する(北海道電力170.93kW vs 市場連動221.56kW).基本料金の差額(約190万円/年)が電力量料金の差額を相殺する.

    \item \textbf{季節変動との相性}:冬季(12--2月)は市場価格が最も低く(平均15.12円/kWh),かつ蓄電池効果が最も高い(買電抑制率64.3\%)季節である.しかし,この時期は需要も低いため,市場価格連動プランのメリットが限定的となる.夏季・秋季は市場価格が高く,かつ蓄電池効果が低いため,市場価格連動プランに不利に働く.
\end{enumerate}

図\ref{fig:price_seasonal_analysis}に市場価格の季節変動と北海道電力基本プランとの比較を示す.

\begin{figure}[H]
    \centering
    \includegraphics[width=0.95\textwidth]{../png/soc860/price_seasonal_analysis.png}
    \caption{市場価格(JEPX)の季節変動分析}
    \label{fig:price_seasonal_analysis}
\end{figure}


\subsection{予測期間と計算コストの妥当性}

シナリオB(予測期間比較)の結果に基づき,予測期間の選択と計算コストのトレードオフを考察する.

市場価格連動プランでは,予測期間を24時間から48時間に延長することで年間コストが約47.6万円(3.0\%)削減された.しかし,48時間から60時間への延長では追加削減がわずか約2.1万円,72時間への延長でも約16.9万円に留まり,\textbf{改善効果の逓減}が確認された(表\ref{tab:horizon_comparison}).

一方,北海道電力基本プランでは予測期間延長の効果がより限定的であり,24時間から72時間への延長で約9.1万円(0.6\%)の削減に留まった(表\ref{tab:horizon_comparison_hokkaido}).これは,電力量単価が一定であるため価格変動への追従が不要であり,買電電力の最大値抑制の精度向上のみが効果として現れるためである.

計算時間は予測期間の延長に伴いほぼ線形に増加した(24時間:12.8分 → 48時間:35.1分 → 60時間:43.7分 → 72時間:58.9分,表\ref{tab:horizon_timing}).費用対効果の観点から,市場価格連動プランでは48時間予測が最も効率的であり,60時間以上への延長は計算コストに見合わない.本システム構成と検証した4つの予測期間(24/48/60/72時間)においては,計算資源と削減効果のバランスから\textbf{48時間(96ステップ)の予測期間が実用的な選択肢の一つ}であると考えられる.

\subsection{年間一括最適化によるローリング計画法の評価}

年間一括最適化との比較の結果から,ローリング計画法の最適性について以下の知見が得られた.

北海道電力基本プランでは,ローリング計画法と年間一括最適化のコスト差は0.28\%~0.69\%と小さく,ローリング計画法でも\textbf{ほぼ最適解}を達成している.電力量料金が一定のため,ソルバが各予測期間で買電電力を平準化すれば,自動的に年間全体での最適運用に近づく.

一方,市場価格連動プランでは8.4\%~18.5\%のコスト差が生じた.原因を以下に示す:

\begin{enumerate}
    \item \textbf{時間的裁定の局所最適化}:ローリング計画法では,各48時間の窓内で「安価な時間帯に買電を集中」という戦略が最適となる.しかし,この局所最適が年間を通して繰り返されると,特定の時間帯(深夜帯など)に買電が集中し,契約電力が増大する.

    \item \textbf{契約電力の増大}:表\ref{tab:annual_vs_rolling_market}に示すように,ローリング計画法では契約電力が201.4〜221.6kWまで増大するのに対し,年間一括最適化では156.4〜185.5kWに抑制される.この差(最大65kW)により,基本料金が約190万円/年増加する.

    \item \textbf{蓄電池容量との関係}:蓄電池容量が大きいほどコスト差率が大きくなる(430kWh: 8.4\%,1290kWh: 18.5\%)傾向は,大容量蓄電池があるほど時間的裁定の余地が大きく,その局所最適化による弊害も大きくなることを示唆する.
\end{enumerate}

\subsubsection{料金プラン優位性への示唆}

本研究の主要な発見の一つである「蓄電池容量540kWh以上で北海道電力基本プランが有利」という結論(表\ref{tab:capacity_plan_comparison})は,ローリング計画法を用いた場合の結果である.年間一括最適化では全容量において市場価格連動プランが有利となった(表\ref{tab:plan_comparison_annual}).

この差異は,ローリング計画法の局所最適化が市場価格連動プランに不利に働くことを示している.言い換えれば,現実の運用においても市場価格変動を「48時間先まで」しか見通せない場合,大容量蓄電池を持つ施設は北海道電力基本プランの方が安全な選択となる可能性がある.一方,より長期の価格予測精度が向上すれば,市場価格連動プランの優位性を活かせる可能性がある.

\subsubsection{長期予測期間による改善効果}

表\ref{tab:long_horizon}の結果は,予測期間の延長が市場価格連動プランの経済性を大幅に改善することを示している.特に,8日間(384ステップ)以上の予測期間では,48時間予測で見られた「北海道電力基本プランが有利」という結論が逆転し,市場価格連動プランが10.6万円安価となった.

\noindent
\textbf{改善メカニズムの分析}:

予測期間延長によるコスト改善は,以下の3つの要因により説明できる:

\begin{enumerate}
    \item \textbf{契約電力の抑制}:48時間予測では市場連動プランの契約電力が218.1kWであったのに対し,8日間予測では193.7kW(▲24.4kW),14日間予測では187.9kW(▲30.2kW),21日間予測では183.3kW(▲34.8kW)まで低下した.これは,長期予測により「今日安価だから買電を増やす」という短期的判断ではなく,「週全体で見て最適なタイミング」を選択できるようになったためである.

    \item \textbf{週単位の価格パターン活用}:JEPXスポット価格は平日に高く週末に安い傾向がある.48時間予測では翌日までしか考慮できないが,8日間以上の予測では週末の低価格時間帯を計画的に活用した時間的裁定が可能となる.

    \item \textbf{年間一括最適化への収束}:予測期間と年間一括最適化との差(年間一括比)は,48時間予測の19.0\%から,8日間で12.5\%,14日間で10.8\%,21日間で9.3\%と逓減的に縮小した.この収束パターンは,予測期間の延長が大域最適への近似精度を改善することを示唆する.
\end{enumerate}

\noindent
\textbf{実用上の示唆}:

21日間予測においても年間一括最適化との差は9.3\%(約117万円)残存しており,ローリング計画法の本質的な限界が示唆される.ただし,14日間あるいは21日間の価格予測を完全に行うことは現実的には不可能であり,予測誤差がコストに与える影響については今後の検証が必要である.現実的には,1週間程度の予測で市場連動プランの優位性を回復できることが重要な知見と言える.

\noindent
\textbf{大容量蓄電池での長期予測効果}:

21日間予測(再計画間隔24)において蓄電池容量1290kWhで実験したところ,北海道電力基本プラン1,400.8万円に対し市場価格連動プラン1,399.1万円と,わずか1.7万円差でほぼ同等となった.48時間予測では540kWh以上で北海道電力基本プランが有利であったが,21日間予測により,大容量蓄電池でも市場連動プランの競争力が回復することが確認された.

\subsection{予測の不確実性に対するロバスト性}
\label{sec:robustness}

本研究のシミュレーションは完全予見(Perfect Forecast)を前提とし,予測期間内の需要・PV発電量・市場価格に誤差がないと仮定している.しかし,実運用では予測誤差が不可避であり,この誤差が最適化結果に与える影響を把握する必要がある.本節では,ローリング計画法の予測誤差に対する頑健性を定性的に分析する.

\subsubsection{フィードバック制御としてのローリング計画法}

ローリング計画法は,制御工学における\textbf{モデル予測制御(Model Predictive Control, MPC)}の一形態であり,以下のフィードバック構造を持つ:

\begin{enumerate}
    \item \textbf{観測}:現時点の系統状態(SOC,需要,PV発電量,市場価格)を取得
    \item \textbf{予測}:将来$H$ステップの状態を予測(本研究では完全予見を仮定)
    \item \textbf{最適化}:予測に基づき$H$ステップの最適制御系列を計算
    \item \textbf{実行}:最初の$m$ステップ(再計画間隔)のみを実行
    \item \textbf{更新}:実績値を反映して(1)に戻る
\end{enumerate}

このフィードバック構造により,ローリング計画法は予測誤差に対してある程度の\textbf{自己修正能力}を持つ.予測が外れても,次の計画サイクルで実績値を観測し,修正された予測に基づいて再計画を行うためである.

\subsubsection{予測誤差の分類と影響の非対称性}

予測誤差が最適化結果に与える影響は,誤差の種類によって大きく異なる.本研究の文脈では,以下の3種類の予測誤差を区別する必要がある:

\paragraph{需要予測誤差}

需要の予測誤差は,蓄電池のSOC管理に直接影響する.しかし,需要は相対的に予測しやすく(日・週のパターンが明確),また誤差が生じても次サイクルで修正可能である.

\begin{itemize}
    \item \textbf{過大予測}:実需要が予測より小さい場合,蓄電池に余裕が生じるが,次サイクルで調整可能
    \item \textbf{過小予測}:実需要が予測より大きい場合,系統からの追加買電で対応可能
\end{itemize}

\paragraph{PV発電量予測誤差}

PV発電量は天候に依存するため,需要より予測が困難である.特に曇天・晴天の境界付近では大きな誤差が生じうる.

\begin{itemize}
    \item \textbf{過大予測}(晴天予測→曇天実績):期待したPV発電が得られず,買電量が増加
    \item \textbf{過小予測}(曇天予測→晴天実績):PV余剰が生じるが,蓄電池で吸収可能(容量に余裕があれば)
\end{itemize}

\paragraph{市場価格予測誤差}

JEPX価格は最も予測が困難であり,特に極端な高騰(スパイク)の予測は現状の技術では限界がある.

\begin{itemize}
    \item \textbf{安価予測→高価実績}:安価と予測した時間帯での買電が,実際には高コストとなる
    \item \textbf{高価予測→安価実績}:買電を回避した結果,安価な機会を逃す
\end{itemize}

\subsubsection{不可逆なコスト構造とフィードバック制御の相性}

ローリング計画法のフィードバック機構は,\textbf{可逆的な状態変数}(SOC,買電量など)に対しては有効に機能する.しかし,\textbf{不可逆なコスト構造}を持つ指標に対しては,フィードバック制御の効果が大きく制限される.

\paragraph{契約電力:「一度の失敗が命取り」の構造}

契約電力(基本料金の決定要因)は,年間を通じた最大買電電力で決定される.この指標は以下の意味で「不可逆」である:

\begin{equation}
P_{\text{contract}} = \max_{t \in \mathcal{Y}} p_t^{\mathrm{BY}}
\end{equation}

一度でも高い買電電力$p_t^{\mathrm{BY}}$を記録すると,それ以降にどれほど低い買電電力を維持しても,契約電力を下げることはできない.これは不可逆過程であり,フィードバック制御による修正が原理的に不可能な構造である.

\paragraph{予測誤差と契約電力増大のリスク}

この不可逆構造は,予測誤差に対する脆弱性を生む.特に以下のシナリオが危険である:

\begin{itemize}
    \item \textbf{需要の突発的急増}:予測を大幅に上回る需要が発生し,蓄電池からの放電だけでは賄えない場合,系統からの大量買電が必要となる
    \item \textbf{PV発電の突発的低下}:晴天予測に基づき蓄電池を温存していたところ,急な曇天でPV発電が激減し,買電が急増
    \item \textbf{市場価格スパイク時の強制買電}:価格が急騰した時間帯に需要ピークが重なり,高価格・高電力での買電を強いられる
\end{itemize}

これらのシナリオでは,「一度の失敗」が年間の基本料金を恒久的に押し上げる.フィードバック制御は「次サイクルでの修正」を前提とするが,契約電力の決定においては「次サイクル」は意味をなさない.

\paragraph{電力量料金:可逆的なコスト構造}

対照的に,電力量料金は可逆的な構造を持つ:

\begin{equation}
C_{\text{energy}} = \sum_{t \in \mathcal{Y}} \pi_t \cdot p_t^{\mathrm{BY}} \cdot \Delta t
\end{equation}

\noindent
ある時点で予測誤差により非効率な買電が発生しても,その影響は当該時点のコスト増加に留まる.別の時点で効率的な運用を行えば,年間総コストへの影響を部分的に相殺できる.この意味で,電力量料金に対してはフィードバック制御が有効に機能する.

\subsubsection{料金プラン別のロバスト性比較}

以上の分析を踏まえ,両料金プランの予測誤差に対するロバスト性を比較する.

\paragraph{北海道電力基本プラン}

北海道電力基本プランは予測誤差に対して頑健である:

\begin{itemize}
    \item \textbf{電力量料金}:単価が一定であるため,「安価な時間帯を逃す」リスクが存在しない
    \item \textbf{契約電力}:買電電力の平準化が常に最適戦略であり,予測誤差が戦略を大きく変えない
    \item \textbf{運用の安定性}:価格変動に依存しないため,運用パターンが予測可能
\end{itemize}

\paragraph{市場価格連動プラン}

市場価格連動プランは,予測誤差に対して脆弱性を持つ:

\begin{itemize}
    \item \textbf{価格予測誤差}:アービトラージ戦略は正確な価格予測を前提とし,誤差が収益を直接毀損
    \item \textbf{需要・PV予測誤差}:価格変動との複合効果により,誤差の影響が増幅されうる
    \item \textbf{契約電力リスク}:安価な時間帯への買電集中戦略は,予測誤差により「意図せぬピーク」を生むリスクを内包
\end{itemize}

特に,本研究で観察された「大容量蓄電池での契約電力増大」現象(表\ref{tab:capacity_comparison_market_detail})は,完全予見下でさえ発生している.予測誤差が加わると,この現象がさらに悪化する可能性が高い.

\subsubsection{実運用への示唆}

本分析から,実運用に向けた以下の示唆が得られる:

\begin{enumerate}
    \item \textbf{安全マージンの導入}:契約電力の不可逆性を考慮し,最適化において明示的な安全マージン(例:予測最大需要の110\%を上限とする制約)を設けることが推奨される

    \item \textbf{ロバスト最適化の検討}:確定的最適化ではなく,予測誤差の確率分布を考慮したロバスト最適化やチャンス制約付き最適化の導入が有効と考えられる

    \item \textbf{プラン選択への反映}:予測精度が低い環境では,市場価格連動プランの理論的優位性が実現しにくい可能性がある.予測誤差を考慮した総合評価では,北海道電力基本プランの相対的優位性が増す可能性がある

    \item \textbf{蓄電池容量の再評価}:第\ref{sec:phase_transition}節で論じた臨界容量(540kWh)は完全予見を前提としている.予測誤差を考慮すると,アービトラージ機能の飽和がより早期に生じ,臨界容量が小さくなる可能性がある
\end{enumerate}

\paragraph{本研究の結論への影響}

本研究の主要結論「北海道電力基本プラン+大容量蓄電池が経済的に有利」は,完全予見を前提としている.予測誤差を考慮すると,この結論は以下の意味で\textbf{保守的(Conservative)}である:

\begin{itemize}
    \item 北海道電力基本プランは予測誤差に対してロバストであり,完全予見下での性能が実運用でも概ね維持される
    \item 市場価格連動プランは予測誤差により性能が劣化する可能性が高く,完全予見下での性能は上界に近い
\end{itemize}

予測誤差を考慮した実運用環境では,北海道電力基本プランの優位性がさらに拡大する可能性がある.この点で,本研究の結論は実運用に対しても一定の妥当性を持つ.

\subsection{モデルの制約と限界}
\label{sec:wbasic_limitation}

本節では,採用した最適化手法の限界と,それが結果に与える影響を明確にする.

\subsubsection{基本料金係数の按分手法}

契約電力(基本料金の決定要因)は,本来「年間17,520ステップの中の最大買電電力」という\textbf{大域的な指標}で決定される.しかし,ローリング計画法では年間を通した最適化が計算上困難であるため,各予測期間(96ステップ,48時間)内の最大値$s^{\mathrm{BY}}_{\mathrm{MAX}}$に按分係数$w_{\mathrm{basic}}$を乗じてペナルティを与える近似を採用した.

この「局所的な最大値の抑制」を通じて「大域的な最大値」を間接的に制御する手法には,以下の数理的問題がある:

\begin{enumerate}
    \item \textbf{最適化基準と評価基準の乖離}:ソルバーは各予測期間で「$w_{\mathrm{basic}} \times s^{\mathrm{BY}}_{\mathrm{MAX}}$」を最小化しようとするが,最終的な基本料金は「年間最大買電電力 × 単価」で計算される.

    \item \textbf{局所最適と大域最適の不一致}:ある予測期間で買電電力の最大値を抑制する努力が,別の期間で発生するより大きな買電電力により無効化される可能性がある.

    \item \textbf{按分係数の重みの影響}:$w_{\mathrm{basic}}$の値によって,ソルバーの「買電電力の最大値抑制」と「電力量料金削減」のトレードオフ判断が変化する.
\end{enumerate}

北海道電力基本プランでは電力量料金が一定であるため,ソルバーは自然に買電電力を平準化する方向に最適化を行い,按分手法の影響は限定的である.一方,市場価格連動プランではJEPX価格の変動(3.80〜31.00円/kWh)により,「安価な時間帯への買電集中」と「買電電力の最大値抑制」のトレードオフが生じ,大容量蓄電池ほど契約電力が増大する傾向が確認された.

\subsubsection{結論の解釈における注意点}

本研究の結論を解釈する際には以下の点に注意が必要である:

\begin{enumerate}
    \item 市場価格連動プランの「最適容量430kWh」は按分係数の設定値に依存する可能性がある
    \item 契約電力の増加現象は,時間的裁定の本質的特性と按分手法の限界の両方が寄与している
    \item 両プラン比較において,按分手法の影響に非対称性がある
\end{enumerate}

\noindent
\textbf{今後の課題}:按分手法の限界を克服するためには,$w_{\mathrm{basic}}$の感度分析,年間契約電力の明示的追跡機構の導入,異なる最適化手法との比較検証が必要である.

\subsubsection{リスクの偏在性と認識の非対称性}
\label{sec:risk_asymmetry}

前節では按分手法の数理的問題を述べたが,本節ではより本質的な問題として「リスクの偏在性」と「認識の非対称性」について論じる.これらの概念は,第\ref{sec:bounded_rationality}節で述べた近視眼的最適化の議論を補完し,市場価格連動プランにおける契約電力増大のメカニズムをより深く理解するための鍵となる.

\paragraph{リスクの時間的偏在性}

本研究で採用した按分係数$w_{\mathrm{basic}}$は,式(\ref{eq:weight_basic})に示す通り,年間を通じて均等に配分されている:

\begin{equation}
w_{\mathrm{basic}} = \frac{\text{基本料金単価} \times H}{17{,}520} = \frac{1989.0 \times 96}{17{,}520} \approx 10.90 \text{ [円/kW]}
\end{equation}

\noindent
この均等配分は,「年間最大買電電力はいつ発生するか分からない」という不確実性への素朴な対応である.しかし,実際のリスク(年間最大買電電力の発生確率)は時間的に均等ではない.

図\ref{fig:monthly_analysis}に示した月別需要パターンから明らかなように,本研究の対象施設では7月・8月の夏季に需要が集中し,特に8月に年間最大買電電力166.8kWを記録した.すなわち,「年間最大買電電力の発生リスク」は夏季の特定期間に偏在している.

この偏在性を定量化すると:

\begin{itemize}
    \item 夏季(6月〜8月):年間需要の約38\%,最大買電電力発生確率が最も高い
    \item 冬季(12月〜2月):年間需要の約15\%,需要の最大電力は大きいがPV発電でカバー可能
    \item 中間期(3月〜5月,9月〜11月):最大買電電力発生確率は低〜中程度
\end{itemize}

均等配分の$w_{\mathrm{basic}}$は,この偏在性を反映していない.結果として,リスクの低い時期(冬季)には過剰なペナルティを,リスクの高い時期(夏季)には不十分なペナルティを課すことになる.

\paragraph{認識の非対称性:「今この瞬間」の特定不能性}

より根本的な問題は,ローリング計画法のエージェントが「今こそが年間最大需要の瞬間である」と認識するメカニズムを持たないことである.

人間の意思決定者であれば,以下の情報を総合して「今日は特に注意が必要」と判断できる:

\begin{itemize}
    \item 過去の経験:「去年も1月上旬が最も需要が高かった」
    \item 気象情報:「明日は今季最強の寒波が到来」
    \item 文脈情報:「今日は大型イベント開催で施設稼働率が最大」
\end{itemize}

一方,本研究のエージェントは48時間の予測期間内のデータのみを参照し,以下の認識の非対称性を抱えている:

\begin{equation}
\underbrace{\text{エージェントの視野}}_{\mathcal{H}: \text{48時間}} \ll \underbrace{\text{リスク評価に必要な視野}}_{\mathcal{Y}: \text{年間}}
\end{equation}

この認識の非対称性により,エージェントは各予測期間において「自分の行動が年間最大買電電力を決定してしまうかもしれない」という危機感を持つことができない.特に市場価格連動プランでは,安価な時間帯(例:深夜帯,2〜4円/kWh)が出現するたびに,エージェントは「この機会を逃すまい」と貪欲に買電を行う.その結果,年間を通じて散発的に発生する「安価だが大量買電」のエピソードのうち,最も大きなものが契約電力を決定することになる.

\paragraph{敗因の構造的理解}

以上の分析を踏まえると,市場価格連動プランにおける契約電力増大の敗因は以下のように構造化できる:

\begin{enumerate}
    \item \textbf{価格シグナルの誘惑}:JEPX価格の変動が「安価な時間帯への買電集中」を誘発
    \item \textbf{大容量蓄電池の買いだめ能力}:蓄電池容量が大きいほど,一時点での大量買電が可能
    \item \textbf{リスク認識の欠如}:均等配分の$w_{\mathrm{basic}}$はリスクの偏在性を反映せず
    \item \textbf{時点識別の不能}:エージェントは「今が年間最大となる瞬間か」を判断できない
    \item \textbf{累積的帰結}:各時点での「局所的に合理的な」大量買電が,年間最大を押し上げる
\end{enumerate}

この構造は,北海道電力基本プランでは発現しない.固定価格は(1)の「価格シグナルの誘惑」を除去し,(2)以降の連鎖を断ち切るからである.

\paragraph{改善への示唆}

リスクの偏在性と認識の非対称性を克服するためのアプローチとして,以下が考えられる:

\begin{itemize}
    \item \textbf{季節別$w_{\mathrm{basic}}$}:冬季には高い$w_{\mathrm{basic}}$,夏季には低い$w_{\mathrm{basic}}$を設定
    \item \textbf{動的$w_{\mathrm{basic}}$}:予測需要が一定閾値を超えた場合に$w_{\mathrm{basic}}$を増大
    \item \textbf{契約電力追跡機構}:現時点までの最大買電電力を状態変数として保持し,それを超える買電に対してのみペナルティを課す
    \item \textbf{確率的定式化}:年間最大需要の発生確率分布を推定し,期待値最小化に基づく最適化
\end{itemize}

これらのアプローチは,エージェントに「大域的視野」を擬似的に付与する試みと解釈できる.ただし,いずれも追加的な計算コストや調整パラメータの導入を伴うため,実装上の課題は残る.

\subsection{一般化可能性と適用範囲}

本研究の結論は,以下の特定条件に依存しており,異なる条件下では結論が変化する可能性がある:

\begin{itemize}
    \item \textbf{対象期間}:2024年1月〜12月のJEPX北海道エリア価格(年間平均16.87円/kWh)
    \item \textbf{地域}:北海道エリア(冬季暖房需要が顕著,本州との連系線容量が限定的)
    \item \textbf{需要特性}:年間812,982kWh,最大電力264.0kW,PV自給率35.4\%
    \item \textbf{システム構成}:PV 250kW,蓄電池出力400kW,売電不可
\end{itemize}

他の電力エリア,異なる年度の市場価格,異なる需要パターン(オフィスビル,工場等),または売電可能なシステム構成では,最適な料金プランや蓄電池容量が変化する.

ただし,以下の\textbf{定性的知見}は一定の普遍性を持つ:

\begin{enumerate}
    \item 料金プランの有利性は蓄電池容量に依存し,逆転する分岐点が存在しうる
    \item 固定価格プランでは契約電力の抑制,変動価格プランでは時間的裁定が最適化の主軸となる
    \item 蓄電池効果は季節により大きく変動する
    \item 予測期間の延長効果は逓減する
\end{enumerate}

異なる条件下での最適解を求めるには,本研究で開発した最適化フレームワークに対象地域の市場価格データ,料金体系,需要パターンを入力して再計算する必要がある.

\section{結論}

本研究では,北海道十勝地方のPV・蓄電池システム(PV容量250kW)を対象に,ローリング計画法による運用最適化シミュレーションを実施し,料金プランと蓄電池容量の関係性を検証した.対象期間は2024年1月1日から12月31日(2月29日を除く365日,17,520ステップ)である.主な結論を以下に示す.

\begin{enumerate}
    \item \textbf{料金プランの優位性と蓄電池容量}:料金プランの経済的優位性は蓄電池容量に依存する.容量430kWh以下では市場価格連動プランが有利,540kWh以上では北海道電力基本プランが有利となり,両プランの優劣が逆転する分岐点が存在する.

    \item \textbf{最適容量における比較}:各プランのコスト最小化を実現する最適容量(市場価格連動プラン:430kWh,北海道電力基本プラン:1720kWh)を比較すると,北海道電力基本プランが年間\textbf{約54万円安価}である.

    \item \textbf{契約電力への影響}:北海道電力基本プランでは蓄電池容量の増加に伴い契約電力が減少する.一方,市場価格連動プランでは安価な時間帯への買電集中により,430kWhを超えると契約電力が増加する.

    \item \textbf{蓄電池導入効果}:蓄電池なし(0kWh)と比較して,北海道電力基本プラン+1720kWhで年間約403万円,市場価格連動プラン+430kWhで年間約281万円のコスト削減を達成した.なお,PV利用率の向上(78\%→100\%)は,蓄電池の物理的な時間シフト機能による寄与が主である.

    \item \textbf{予測期間の妥当性}:予測期間の延長はコスト削減に寄与するが,その効果は逓減する.計算コストと削減効果のトレードオフを考慮すると,本システム構成では48時間(96ステップ)の予測期間が実用的である.

    \item \textbf{年間一括最適化との比較}:ローリング計画法と年間一括最適化の比較により,北海道電力基本プランではローリング計画法でもほぼ最適解(コスト差0.3~0.7\%)を達成するが,市場価格連動プランでは8~18\%のコスト差が生じる.これは時間的裁定の局所最適化が契約電力増大を招くためである.

    \item \textbf{長期予測期間と再計画間隔}:再計画間隔を4時間(8ステップ)に延長することで計算時間を約88\%削減しつつ,解の品質への影響は0.1\%未満に抑制できた.また,予測期間を14日間(672ステップ)まで延長することで,市場価格連動プランの年間一括最適化との差が22\%から11\%に縮小し,8日間以上の予測では市場連動プランの優位性が回復した.
\end{enumerate}

\end{document}
