\documentclass[a4paper,11pt,dvipdfmx]{jsarticle}

\usepackage[top=20mm, bottom=25mm, left=20mm, right=20mm]{geometry}
\usepackage{url}
\usepackage{graphicx}
\usepackage{amssymb}
\usepackage{amsmath}
\usepackage{booktabs}
\usepackage{float}
\usepackage{multirow}
\usepackage{tikz}
\usetikzlibrary{shapes,arrows,positioning}

\title{ローリング計画法を用いたPV・蓄電池システムの\\運用最適化と電気料金削減効果\\[0.5em]\large 今週の進捗報告}
\author{神戸大学工学部情報知能工学科4年\\山崎博之}
\date{2026年1月21日}

\begin{document}

\maketitle

\section{今週の取り組み概要}

今週は以下の3つの実験・分析を実施した.

\begin{enumerate}
    \item \textbf{年間一括最適化(ローリングを使わない手法)}:年間全データ(17,520ステップ)を一括で最適化し,ローリング計画法との理論的限界を比較
    \item \textbf{制御ホライズンの延長}:ローリング計画法において,計画を更新するステップ間隔を1ステップ(30分)から8ステップ(4時間)に拡大し,計算時間への影響を検証
    \item \textbf{長期予測期間の検証}:予測期間を48時間から最大14日間(672ステップ)まで延長し,市場価格連動プランの経済性改善を検証
\end{enumerate}

\section{年間一括最適化の結果}

\subsection{ローリング計画法との比較}

年間全データを一括で最適化する最適化計画を実行し,ローリング計画法(48時間予測)と比較した.

\begin{table}[H]
\centering
\caption{ローリング計画法 vs 年間一括最適化(蓄電池860kWh)}
\begin{tabular}{lrrrr}
\toprule
手法 & 北電基本 [万円] & 市場連動 [万円] & 計算時間 [分] \\
\midrule
ローリング(48時間予測)& 1,434.4 & 1,502.4 & 33.2 \\
年間一括最適化 & 1,427.8 & 1,262.9 & 4.5 \\
\midrule
差額 & 6.6 (0.5\%) & 239.5 (15.9\%) & - \\
\bottomrule
\end{tabular}
\end{table}

\noindent
\textbf{主な発見}:
\begin{itemize}
    \item 北海道電力基本プランでは,ローリング計画法でもほぼ最適解(差0.5\%)を達成
    \item 市場価格連動プランでは,ローリング計画法と年間一括最適化で\textbf{約240万円}(15.9\%)の差
    \item 年間一括最適化の計算時間は約5分と,ローリング計画法(33分)より約7倍高速
\end{itemize}

\subsection{料金プラン優位性の逆転}

\begin{table}[H]
\centering
\caption{最適化手法による料金プラン優位性の違い}
\begin{tabular}{lrrl}
\toprule
最適化手法 & 北電基本 [万円] & 市場連動 [万円] & 有利なプラン \\
\midrule
ローリング(48時間)& 1,434.4 & 1,502.4 & 北電基本(68万円差)\\
年間一括最適化 & 1,427.8 & 1,262.9 & \textbf{市場連動(165万円差)}\\
\bottomrule
\end{tabular}
\end{table}

\noindent
年間一括最適化では市場価格連動プランが\textbf{165万円安価}であり,48時間ローリングでの結論「北電基本が有利」と逆転する.これは,ローリング計画法において市場連動プランが局所最適に陥っていることを示唆する.

\section{制御ホライズンの影響}

制御ホライズンとは,1回の最適化計算で採用する決定ステップ数である.

\begin{table}[H]
\centering
\caption{制御ホライズンの影響(予測期間48時間,蓄電池860kWh)}
\begin{tabular}{lrrrr}
\toprule
制御ホライズン & 北電基本 [万円] & 市場連動 [万円] & 計算時間 [分] & 時間削減率 \\
\midrule
1ステップ(30分)& 1,434.4 & 1,502.4 & 33.2 & - \\
2ステップ(1時間)& 1,434.4 & 1,502.2 & 15.9 & 52\% \\
4ステップ(2時間)& 1,434.4 & 1,501.6 & 7.9 & 76\% \\
8ステップ(4時間)& 1,434.4 & 1,500.7 & 4.1 & \textbf{88\%} \\
\bottomrule
\end{tabular}
\end{table}

\noindent
\textbf{主な発見}:
\begin{itemize}
    \item 制御ホライズン8で計算時間を\textbf{88\%削減}(33分→4分)
    \item コストへの影響は最大0.1\%(約0.2万円)と極めて限定的
    \item 計算効率の大幅改善と解品質の維持を両立できることを確認
\end{itemize}

\section{長期予測期間の効果}

制御ホライズンの延長により計算時間を抑制しつつ,予測期間を最大14日間まで延長した.

\begin{table}[H]
\centering
\caption{長期予測期間と制御ホライズンの組み合わせ(蓄電池860kWh)}
\begin{tabular}{llrrrr}
\toprule
予測期間 & 制御H & 北電基本 [万円] & 市場連動 [万円] & 計算時間 [分] & 年間一括比 \\
\midrule
48時間 & 1 & 1,434.4 & 1,502.4 & 33.2 & 19.0\% \\
\midrule
4日間 & 4 & 1,433.1 & 1,452.5 & 23.1 & 15.0\% \\
4日間 & 8 & 1,433.1 & 1,448.3 & 11.1 & 14.7\% \\
\midrule
8日間 & 4 & 1,431.5 & 1,420.9 & 53.1 & 12.5\% \\
8日間 & 8 & 1,431.5 & 1,420.6 & 25.5 & 12.5\% \\
\midrule
14日間 & 8 & 1,430.4 & 1,398.9 & 48.5 & 10.8\% \\
14日間 & 16 & 1,430.4 & 1,398.8 & 25.1 & 10.8\% \\
\midrule
21日間 & 24 & 1,429.9 & \textbf{1,380.2} & 27.6 & \textbf{9.3\%} \\
\midrule
\multicolumn{2}{l}{年間一括最適化} & 1,427.8 & 1,262.9 & 4.5 & 0\% \\
\bottomrule
\end{tabular}
\end{table}

\noindent
「年間一括比」は市場価格連動プランにおいて,年間一括最適化との差を示す.

\noindent
\textbf{主な発見}:
\begin{itemize}
    \item 14日間予測で市場連動プランのコストが\textbf{1,398.9万円}まで改善(48時間比:▲103.5万円)
    \item 年間一括最適化との差が22.0\%から10.8\%に縮小
    \item \textbf{8日間以上の予測期間で市場連動プランの優位性が回復}(北電基本より10.6万円安価)
    \item 14日間予測でも制御ホライズン8を用いれば計算時間は約50分に抑制可能
\end{itemize}

\subsection{長期予測期間と蓄電池容量の組み合わせ}

8日間予測(制御ホライズン8)において,蓄電池容量を変化させた結果を表に示す.

\begin{table}[H]
\centering
\caption{8日間予測(h384\_c8)における蓄電池容量別の比較}
\begin{tabular}{rrrrrl}
\toprule
容量 [kWh] & 北電基本 [万円] & 市場連動 [万円] & 差額 [万円] & 計算時間 [分] & 有利なプラン \\
\midrule
430 & 1,486.4 & 1,402.6 & $-$83.8 & 105.3 & \textbf{市場連動} \\
860 & 1,431.5 & 1,420.6 & $-$10.9 & 25.5 & \textbf{市場連動} \\
1290 & 1,402.9 & 1,452.7 & $+$49.8 & 35.2 & 北電基本 \\
1720 & 1,375.6 & 1,452.5 & $+$76.9 & 37.5 & 北電基本 \\
\bottomrule
\end{tabular}
\end{table}

\noindent
\textbf{主な発見}:
\begin{itemize}
    \item 8日間予測では,蓄電池容量860kWh以下で市場連動プランが有利
    \item 48時間予測では540kWh以上で北電基本が有利だったが,長期予測により分岐点が860kWh付近まで移動
    \item 契約電力も改善:860kWhで市場連動プランの契約電力が193.7kW(48時間予測の218.1kWから▲24.4kW)
\end{itemize}

\section{まとめ}

\begin{enumerate}
    \item \textbf{年間一括最適化を実装}:ローリング計画法の理論的限界を定量評価するベンチマークを構築
    \item \textbf{ローリング計画法の限界を発見}:市場連動プランで15.9\%の非効率性,料金プラン優位性が逆転する現象を確認
    \item \textbf{制御ホライズン延長の有効性}:計算時間88\%削減しつつコスト影響0.1\%未満を確認
    \item \textbf{長期予測期間の効果}:14日間予測で年間一括との差を約半減,8日間以上で市場連動プランの優位性回復
\end{enumerate}

\end{document}
